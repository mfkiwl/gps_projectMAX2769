\section{БЖД}

\subsection{Анализ пожарной опасности в помещении небольших размеров, где установлена вычислительная техника}

В соответствии со СНиП 2280 все производства делят по пожарной, взрывной и взрывопожарной опасности на 6 категорий.
А взрывопожароопасные: производства, в которых применяют горючие газы с нижним пределом воспламенения 10 и ниже,
жидкости с tвсп 280 C при условии, что газы и жидкости могут образовывать взрывоопасные смеси в объеме, превышающем
5 объема помещения, а также вещества которые способны взрываться и гореть при взаимодействии с водой, кислородом
воздуха или друг с другом (окрасочные цехи, цехи с наличием горючих газов и тому подобное). Б взрывопожароопасные:
производства, в которых применяют горючие газы с нижним пределом воспламенения выше 10; жидкости tвсп = 28...610С
включительно; горючие пыли и волокна, нижний концентрационный предел воспламенения которых 65 Г/м3 и ниже, при условии,
что газы и жидкости могут образовывать взрывоопасные смеси в объеме, превышающем 5 объема помещения (аммиак,
древесная пыль). В пожароопасные: производства, в которых применяются горючие жидкости с tвсп 610С и горючие пыли или
волокна с нижним пределом воспламенения более 65 Г/м3, твердые сгораемые материалы, способные гореть, но не
взрываться в контакте с воздухом, водой или друг с другом. Г производства, в которых используются негорючие
вещества и материалы в горячем, раскаленном или расплавленном состоянии, а также твердые вещества, жидкости или
газы, которые сжигаются в качестве топлива. Д производства, в которых обрабатываются негорючие вещества и материалы
в холодном состоянии (цехи холодной обработки материалов и так далее). Е взрывоопасные: производства, в которых
применяют взрывоопасные вещества (горючие газы без жидкостной фазы и взрывоопасные пыли) в таком количестве при
котором могут образовываться взрывоопасные смеси в объеме превышающем 5 объема помещения, и в котором по условиям
технологического процесса возможен только взрыв (без последующего горения); вещества, способные взрываться
(без последующего горения) при взаимодействии с водой, кислородом воздуха или друг с другом. Правила устройства
электроустановок ПУЭ регламентируют устройство электрооборудования в промышленных помещениях и для наружных
технологических установок на основе классификации взрывоопасных зон и смесей. Зона класса В. Помещения,
в которых могут образовываться взрывоопасные смеси паров и газов с воздухом при нормальных условиях работы
(слив ЛВЖ в открытые сосуды). Зона класса Ва. Взрывоопасные смеси не образуются при нормальных условиях
эксплуатации оборудования, но могут образоваться при авариях и неисправностях. Зона класса Вб: а) помещения,
в которых находятся горючие газы и пары с высоким нижним пределом воспламенения (15 и более) с резким запахом
(аммиак); б) помещения, в которых могут образовываться взрывоопасные смеси в объеме превышающем 5 объема
помещения. Зона класса Вв. Наружные установки, в которых находятся взрывоопасные газы, пары и ЛВЖ. Зона класса
В. Обработка горючих пылей и волокон, которые могут образовать взрывоопасные смеси при
нормальном режиме работы. Зона класса Ва. В при авариях или неисправностях. Помещения и установки, в которых
содержатся ГЖ и горючие пыли с нижним концентрационным пределом выше 65 Г/м3, относят к пожароопасным и классифицируют.
Зона класса П . Помещения, в которых содержатся ГЖ. Зона класса П . Помещения, в которых содержатся горючие пыли
с нижним концентрационным пределом выше 65 Г/м3. Зона класса П а. Помещения, в которых содержатся твердые горючие
вещества, не способные переходить во взвешенном состояние. Установки класса П . Наружные установки, в которых
содержатся ГЖ (tвосп 610С) и твердые горючие вещества.

В современной промышленности все шире и шире используется вычислительная техника . Работа сотрудников вычислительных
центров (программистов ,операторов, технических работников) при решении производственных задач сопровождается
активизацией внимания и других психологических функций. Все сотрудники ВЧ подвергаются воздействию вредных и
опасных факторов производственной среды таких как электромагнитное поле, статическая электроэнергия, шум, вибрация,
недостаточное освещение и психоэмоциональное напряжение. Особенности характера и режима роботы, значительное
умственное напряжение приводят к изменению у работников ВЦ функционального состояния центральной нервной системы,
нервно мышечного аппарата рук при работе с клавиатурой . Нерациональные конструкция и размещение элементов рабочего
места вызывают необходимость поддержки неудовлетворительной рабочей позы.Длительный дискомфорт приводит к увеличению
напряжения мышц и обуславливает развитие общей усталости и снижение работоспособности . При длительной работе за
экраном монитора значительно напрягается зрительный аппарат с появлением жалоб на головную боль, раздражительность,
нарушение сна, усталость и болезненные ощущения в глазах, пояснице, в области шеи, рук . Для предотвращения
неблагоприятного воздействия на человека вредных факторов, сопровождающих работу с видеодисплейными терминалами
и персональными электронно-вычислительными машинами разработан ряд санитарно-гигиенические требований.
Производственные помещения должны проектироваться в соответствии к требования м СНиП 2.09.04.87
“Административные и бытовые помещения и строения промышленных предприятий ” и СНиП 512-78 - “Инструкция проектирования
строений и помещений для електроновычислительных машин”. Помещения для ЭВМ размещать в подвалах не допускается.
Дверные проходы внутренних помещений должны быть без порогов .При разных уровнях пола соседних помещений в местах
перехода необходимо устанавливать наклонные плоскости (пандусы). Поверхность пола в помещениях эксплэксплуатации ВДТ
и ПЭВМ должна быта, ровной, без выбоин, нескользкой, удобной для очистки и влажной уборки, обладать антистатическими
свойствами. Для внутренней отделки интерьера, должны использоваться диффузно-отражающие материалы с коэффициентом
отражения для потолка - 0,7-0,8; для стен - 0,5-0,6; для пола-0,3-0,5 , они также должны быть разрешены для применения
органами и учреждениями Государственного санитарно эпидемиологического надзора. Вычислительные машины устанавливаются
и размещаются согласно требованиям завода изготовителя и документации. Рабочие места операторов ЭВМ необходимо
размещать с противоположной стороны шумных агрегатов вычислительных машин ; они должны иметь естественное и искусcтвенное
освещение. Площадь на одно рабочее место должна быть не менее 6,0 кв. м, а объем - не менее 24,0 куб.м. с учетом
максимального числа одновременно работающих в смене. Схемы размещения рабочих мест с ВДТ и ПЭВМ должны учитывать
расстояния между рабочими столами с видеомониторами (в направлении тыла поверхности одного видеомонитора и экрана
другого видеомонитора), которое должно быть не менее 2,0 м, а расстояние между боковыми поверхностями
видеомониторов - не менее 1,2 м. Рабочие места с ВДТ и ПЭВМ в залах электронно-вычислительных машин или в помещениях
с источниками вредных производственных факторов должны размешаться в изолированных кабинах с организованным
воздухообменом. Производственные помещения, в которых для работы используются преимущественно ВДТ и ПЭВМ
(диспетчерские, операторские, расчетные и др.) не должны граничить с помещениями, в которых уровни шума и вибрации
превышают нормируемые значения (механические цеха, мастерские и т.п.). Шкафы, сейфы, стеллажи для хранения дисков,
дискет, комплектующих деталей, запасных блоков ВДТ и ПЭВМ, инструментов, следует располагать в подсобных помещениях.
Конструкция рабочего стола должна обеспечивать оптимальное размещение на рабочей поверхности используемого оборудования с
учетом его количества и конструктивных особенностей (размер ВДТ и ПЭВМ, клавиатуры, пюпитра и др.), характера выполняемой
работа. При этом допускается использование рабочих столов различных конструкций, отвечающих современным требованиям
эргономики. Конструкция рабочего стула (кресла) должна обеспечивать поддержание рациональной рабочей позы при работе на
ВДТ и ПЭВМ, позволять изменять позу с целью снижения статического напряжения мышц шейно-плечевой области и спины для
предупреждения развития утомления. Тип рабочего стула (кресла) должен выбираться в зависимости от характера и
продолжительности работы с ВДТ и ПЭВМ с учетом роста пользователя. Рабочий стул (кресло) должен быть
подъемно-поворотными регулируемым по высоте и углам наклона сиденья и спинки, а также расстоянию спинки от переднего
края сиденья, при этом регулировка каждого параметра должна быть независимой, легко осуществляемой и иметь надежную
фиксацию Помещения с ВДТ и ПЭВМ должны оборудоваться системами отопления, кондиционирования воздуха или эффективной
приточно-вытяжной вентиляцией. Расчет воздухообмена следует проводитъ по теплоизбыткам от машин, людей, солнечной
радиации и искусственного освещения. Требования к вентиляции , отоплению и кондиционированию воздуха в ВЦ выполняются
согласно раздела СниП II 37 75 “Отопление , вентиляция и кондиционирование воздуха” . В помещениях с превышенным уровнем
тепла необходимо предвидеть регулировку подачи теплоносителя для выполнения нормативных параметров теплоносителя.
Как обогревательные устройства в машинных залах и архивах информации необходимо
устанавливать регистры из гладких труб или панелей излучающего отопления .Нельзя использовать водонагревательные устройства и паровое отопление в архивах магнитных носителей информации , а также в машинных залах . Воздух , который поступает в помещения ВЧ , следует очищать от загрязнения , в том числе от пыли и микроорганизмов . Параметры микроклимата должны быть следующими : - в холодный период года : температура воздуха 22 ... 24 C ; относительная влажность 60 … 40\% ; - в теплый период года: температура воздуха 21.. 25 C ; относительная влажность 60 … 40\% . Для повышения влажности воздуха в помещениях с ВДТ и ПЭВМ следует применять увлажнители воздуха, заправляемые ежедневно дистиллированной или прокипяченной питьевой водой. Допустимый уровень звукового давления , звука и эквивалентные уровни звука на рабочих местах должны отвечать требованиям “ Санитарных допустимых норм уровней шумов на рабочих местах ” № 3223-85. Для уменьшения шума и вибраций в помещениях ВЦ оборудование и приборы необходимо устанавливать на специальные фундаменты и амортизирующие прокладки , описанные в нормативных документах. Снизить уровень шума в помещениях с ВДТ и ПЭВМ можно также использованием звукопоглощающих материалов с максимальными коэффициентами звукопоглощения в области частот 63 - 8000 Гц для отделки помещений (разрешенных органами и учреждениями Госсанэпиднадзора), подтвержденных специальными акустическими расчетами. Дополнительным звукопоглощением служат однотонные занавеси из плотной ткани, гармонирующие с окраской стен и подвешенные в складку на расстоянии 15-20 см от ограждения. Ширина занавеси должна быть в 2 раза больше ширины окна. Шумящее оборудование (АЦПУ, принтеры и т.п.), уровни шума которого превышают нормированные, должно находиться вне помещения с ВДТ и ПЭВМ. При выполнении основной работы на ВД'Т и ПЭВМ (диспетчерские, операторские, расчетные кабины и посты управления, залы вычислительной техники и др.) в помещениях с ВДТ и ПЭВМ уровень шума на рабочем месте не должен превышать 50 дБ (А). В помещениях, где работают инженерно-технические работники, осуществляющие лабораторный, аналитический или измерительный контроль, уровень шума не должен превышать 60 дБ (А). В помещениях операторов ЭВМ (без дисплеев) уровень шума не должен превышать 65 дБ (А). На рабочих местах в помещениях для размещения шумных агрегатов вычислительных машин (АЦПУ, принтеры и т.п.) уровень шума не должен превышать 75 дБ (А) . Вибрация оборудования на рабочих местах не должна превышать допустимых величин , установленных “Санитарными нормами вибрации рабочих мест” № 3044 84 . Освещение в помещениях ВЦ должно быть смешанным (естественное и искусственное ). Рабочие места с ВДТ и ПЭВМ по отношению к световым проемам должны располагаться так, чтобы естественный свет падал сбоку, преимущественно слева. Искусственное освещение в помещениях эксплуатации ВДТ и ПЭВМ должно осуществляться системой общего равномерного освещения. В производственных и административно-общественных помещениях, в случаях преимущественной работы с документами, допускается применение системы комбинированного освещения (к общему освещению дополнительно устанавливаются светильники местного освещения, предназначенные для освещения зоны расположения документов). Освещенность на поверхности стола в зоне размещения рабочего документа должна быть 300-500 лк. Допускается установка светильников местного освещения для подсветки документов. Местное освещение не должно создавать бликов на поверхности экрана и увеличивать освещенность экрана более 300 лк. Следует ограничивать прямую блесткость от источников освещения, при этом яркость светящихся поверхностей (окна, светильники и др.), находящихся в поле зрения, должна быть не более 200 кд/ кв.м. Следует ограничивать отраженную блесткость на рабочих поверхностях (экран, стол, клавиатура и др.) за счет правильного выбора типов светильников и расположения рабочих мест по отношению к источникам естественного и искусственного освещения, при этом яркость бликов на экране ВДТ и ПЭВМ не должна превышать 40 кд/кв.м и яркость потолка, при применении системы отраженного освещения, не должна превышать 200 кд/кв.м. Показатель ослеплености для источников общего искусственного освещения в производственных помещениях должен быть не более 20, показатель дискомфорта в административно-общественных помещениях не более 40, в дошкольных и учебных помещениях не более 25. Следует ограничивать неравномерность распределения яркости в поле зрения пользователя ВДТ и ПЭВМ, при этом соотношение яркости между рабочими поверхностями не должно превышать 3:1-5:1, а между рабочими поверхностями и поверхностями стен и оборудования 10:1. В качестве источников света при искусственном освещении должны применяться преимущественно люминесцентные лампы типа ЛБ. При устройстве отраженного освещения в производственных и административно-общественных помещениях допускается применение металлогалогенных ламп мощностью до 250 Вт. Допускается применение ламп накаливания в светильниках местного освещения. Общее освещение следует выполнять в виде сплошных или прерывистых линий светильников, расположенных сбоку от рабочих мест, параллельно линии зрения пользователя при рядном расположении ВДТ и ПЭВМ. При периметральном расположении компьютеров линии светильников должны располагаться локализованно над рабочим столом ближе к его переднему краю, обращенному к оператору. Для освещения помещений с ВДТ и ПЭВМ следует применять светильники серии ЛПОЗ6 с зеркализованными решетками, укомплектованные высокочастотными пускорегулирующими аппаратами (ВЧ ПРА). Применение светильников без рассеивателей и экранирующих решеток не допускается. Яркость светильников общего освещения в зоне углов излучения от 50 до 90 градусов с вертикалью в продольной и поперечной плоскостях должна составлять не более 200 кд/кв.м, защитный угол светильников должен быть не менее 40 градусов. Светильники местного освещения должны иметь не просвечивающий отражатель с защитным углом не менее 40 градусов. Коэффициент запаса (Кз) для осветительных установок общего освещения должен приниматься равным 1,4. Коэффициент пульсации не должен превышать 5\%, что должно обеспечиваться применением газоразрядных ламп в светильниках общего и местного освещения с высокочастотными пускорегулирующими аппаратами (ВЧ ПРА) для любых типов светильников. При отсутствии светильников с ВЧ ПРА лампы многоламповых светильников или рядом расположенные светильники общего освещения следует включать на разные фазы трехфазной сети. Для обеспечения нормируемых значений освещенности в помещениях использования ВДТ и ПЭВМ следует проводить чистку стекол оконных рам и светильников не реже двух раз в год и проводить своевременную замену перегоревших ламп. Для предотвращения образования статической электроэнергии и защиты от нее в помещениях ВЦ необходимо использовать нейтрализаторы. Защиту от статического электричества необходимо проводить в соответствии с санитарно
гигиеническими нормами допустимого напряжения электрического поля .Допустимый уровень напряжения электростатических полей не должен превышать 20 Вт втечении одного часа. Оборудование визуального отображения генерирует несколько типов излучения , в том числе рентгеновское , радиочастотное , ультрафиолетовое , но уровни этих излучений достаточно низкие и не превышают норм. В машинных залах ЭВМ и помещениях с ВДТ необходимо контролировать уровень аэроионизации . Необходимо учитывать , что мягкое рентгеновское излучение , которое возникает при напряжении на аноде монитора 20…22 кВ , а также высокое напряжение на токоведущих участках схем вызывают ионизацию воздуха с созданием позитивных ионов , которые считаются вредными для человека . Оптимальным уровнем аэроионизации в зоне дыхания работающего считается содержание легких аэроионов обоих знаков от 0,015 до 0,00015 в 1 см.куб. воздуха. Режимы труда и отдыха при работе с ПЭВМ и ВДТ должны организовываться в зависимости от вида и категории трудовой деятельности. Виды трудовой деятельности разделяются на 3 группы: группа А - работа по считыванию информации с экрана ВДТ или ПЭВМ с предварительным запросом: группа Б - работа по вводу информации; группа В - творческая работа в режиме диалога с ЭВМ. При выполнении в течение рабочей смены работ, относящихся к разным видам трудовой деятельности, за основную работу с ПЭВМ и ВДТ следует принимать такую, которая занимает не менее 50\% времени в течение рабочей смены или рабочего дня. Для видов трудовой деятельности устанавливается 3 категории тяжести и напряженности работы с ВДТ и ПЭВМ которые определяются: для группы А - по суммарному числу считываемых знаков за рабочую смену, но не более 60 000 знаков за смену; для группы Б - по суммарному числу считываемых или вводимых знаков за рабочую смену, но не более 40 000 знаков за смену; для группы В - по суммарному времени непосредственной работы с ВДТ и ПЭВМ за рабочую смену, но не более 6 часов за смену. Продолжительность обеденного перерыва определяется действующим законодательством о труде и Правилами внутреннего трудового распорядка предприятия (организации, учреждения). Для обеспечения оптимальной работоспособности и сохранения здоровья профессиональных пользователей, на протяжении рабочей смены должны устанавливаться регламентированные перерывы. Время регламентированных перерывов в течение рабочей смены следует устанавливать, в зависимости от ее продолжительности, вида и категории трудовой деятельности. Продолжительность непрерывной работы с ВДТ без регламентированного перерыва не должна превышать 2 часов. При работе с ВДТ и ПЭВМ в ночную смену (с 22 до 6 часов), независимо от категории и вида трудовой деятельности, продолжительность регламентированных перерывов должна увеличиваться на 60 минут. При 8-ми часовой рабочей смене и работе на ВДТ и ПЭВМ регламентированные перерывы следует устанавливать: - для 1 категории работ через 2 часа от начала рабочей смены и через 2 часа после обеденного перерыва продолжительностью 15 минут каждый; - для 11 категории работ через 2 часа от начала рабочей смены и через 1,5-2,0 часа после обеденного перерыва продолжительностью 15 минут каждый или продолжительностью 10 минут через каждый час работы; - для III категории работ через 1,5-2,0 часа от начала рабочей смены и через 1.5-2 часа после обеденного перерыва продолжительностью 20 минут каждый или продолжительностью 15 минут через каждый час работы. При 12-ти часовой рабочей смене р егламентированные перерывы должны устанавливаться в первые 8 часов работы аналогично перерывам при 8-ми часовой рабочей смене, а в течение последних 4часов работы, независимо от категории и вида работ, каждый час продолжительностью 15 минут. Во время регламентированных перерывов с целью снижения нервно-эмоционального напряжения, утомления зрительного анализатора, устранения влияния гиподинамии и гипокинезии, предотвращения развития познотонического утомления целесообразно выполнять комплексы специальных упражнений. С целью уменьшения отрицательного влияния монотонии целесообразно применять чередование операций осмысленного текста и числовых данных (изменение содержания работ), чередование редактирования текстов и ввода данных (изменение содержания работы). В случаях возникновения у работающих с ВДТ и ПЭВМ зрительного дискомфорта и других неблагоприятных субъективных ощущений, несмотря на соблюдение санитарно-гигиенических, эргономических требований, режимов труда и отдыха следует применять индивидуальный подход в ограничении времени работ с ВДТ и ПЭВМ коррекцию длительности перерывов для отдыха или проводить смену деятельности на другую, не связанную с использованием ВДТ и ПЭВМ. Работающим на ВДТ и ПЭВМ с высоким уровнем напряженности во время регламентированных перерывов и в конце рабочего дня показана психологическая разгрузка в специально оборудованных помещениях (комната психологической разгрузки). Для предупреждения развития переутомления обязательными мероприятиями являются: - проведение упражнений для глаз через каждые 20-25 минут работы за ВДТ и ПЭВМ - подключение таймера к ВДТ и ПЭВМ или централизованное отключение свечения информации на экранах видеомониторов с целью обеспечения нормируемого времени работы на ВДТ или ПЭВМ; - проведение во время перерывов сквозного проветривания помещений с ВДТ или ПЭВМ ; - осуществление во время перерывов упражнений физкультурной паузы в течение 3-4 минут); - проведение упражнений физкультминутки в течение 1-2 минут для снятия локального утомления, которые должны выполняться индивидуально при появлении начальных признаков усталости; - замена комплексов упражнений один раз в 2-3 недели. Использованная литература

\subsection{Оснащение помещения устойством для локального тушения пожаров}

\subsection{Экологическая оценка проектируемой компьютерной техники}

\newpage

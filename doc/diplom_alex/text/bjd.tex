\section{БЖД}

\subsection{Анализ пожарной опасности в помещении небольших размеров, где установлена вычислительная техника}

Пожары в ВЦ представляют особую опасность, так как сопряжены с большими материальными потерями.
Характерная особенность ВЦ - небольшие площади помещений. Как известно пожар может возникнуть при взаимодействии
горючих веществ, окисления и источников зажигания. В помещениях ВЦ присутствуют все три основные фактора,
необходимые для возникновения пожара. Горючими компонентами на ВЦ являются: строительные материалы для
акустической и эстетической отделки помещений, перегородки, двери, полы, перфокарты и перфоленты, изоляция кабелей и др.

Противопожарная защита - это комплекс организационных и технических мероприятий, направленных на обеспечение безопасности людей, на предотвращение пожара, ограничение его распространения, а также на создание условий для успешного тушения пожара.

Источниками зажигания в ВЦ могут быть электронные схемы от ЭВМ, приборы, применяемые для технического обслуживания, устройства электропитания, кондиционирования воздуха, где в результате различных нарушений образуются перегретые элементы, электрические искры и дуги, способные вызвать загорания горючих материалов.

В современных ЭВМ очень высокая плотность размещения элементов электронных схем. В непосредственной близости
друг от друга располагаются соединительные провода, кабели. При протекании по ним электрического тока выделяется
значительное количество теплоты. При этом возможно оплавление изоляции. Для отвода избыточной теплоты от ЭВМ служа
системы вентиляции и кондиционирования воздуха. При постоянном действии эти системы представляют собой дополнительную
пожарную опасность. Энергоснабжение ВЦ осуществляется от трансформаторной станции и двигатель-генераторных агрегатов.
На трасформаторных подстанциях особую опасность представляют трансформаторы с масляным охлаждением. В связи с этим
предпочтение следует отдавать сухим транформатором. Пожарная опасность двигатель-генераторных агрегатов обусловленна
возможностью коротких замыканий, перегрузки, электрического искрения. Для безопасной работы необходим правильный
расчет и выбор аппаратов защиты. При поведении обслуживающих, ремонтных и профилактических работ используются
различные смазочные вещества, легковоспламеняющиеся жидкости, прокладываются временные электропроводники, ведут пайку
и чистку отдельных узлов. Возникает дополнительная пожарная опасность, требующая дополнительных мер пожарной защиты.
В частности, при работе с паяльником следует использовать несгораемую подставку с несложными приспособлениями
для уменьшения потребляемой мощности в нерабочем состоянии.

Для большинства помещений ВЦ установлена категория пожарной опасности. Одной из наиболее важных задач пожарной защиты
является защита строительных помещений от разрушений и обеспечение их достаточной прочности в условиях воздействия
высоких температур при пожаре. Учитывая высокую стоимость электронного оборудования ВЦ, а также категорию его
пожарной опасности, здания для ВЦ и части здания другого назначения, в которых предусмотрено размещение ЭВМ должны
быть 1 и 2 степени огнестойкости. Для изготовления строительных конструкций используются, как правило, кирпич,
железобетон, стекло, металл и другие негорючие материалы. Применение дерева должно быть ограниченно, а в
случае использования необходимо пропитывать его огнезащитными составами. В ВЦ противопожарные преграды в виде
перегородок из несгораемых материалов устанавливают между машинными залами.

К средствам тушения пожара, предназначенных для локализации небольших загораний, относятся пожарные стволы,
внутренние пожарные водопроводы, огнетушители, сухой песок, асбестовые одеяла и т. п. В зданиях ВЦ пожарные
краны устанавливаются в коридорах, на площадках лестничных клеток и входов. Вода используется для тушения пожаров
в помещениях программистов, библиотеках, вспомогательных и служебных помещениях. Применение воды в машинных залах
ЭВМ, хранилищах носителей информации, помещениях контрольно измерительных приборов ввиду опасности повреждения или
полного выхода из строя дорогостоящего оборудования возможно в исключительных случаях, когда пожар принимает
угрожающе крупные размеры. При этом количество воды должно быть минимальным, а устройства ЭВМ необходимо защитить
от попадания воды, накрывая их бризентом или полотном.

Для тушения пожаров на начальных стадиях широко применяются огнетушители. По виду используемого огнетушащего
вешества огнетушители подразделяются на следующие основные группы. Пенные огнетушители, применяются для тушения
горящих жидкостей, различных материалов, конструктивных элементов и оборудования, кроме электрооборудования,
находящегося под напряжением. Газовые огнетушители применяются для тушения жидких и твердых веществ, а также
электроустановок, находящихся под напряжением. В производственных помещениях ВЦ применяются главным образом
углекислотные огнетушители, достоинством которых является высокая эффективность тушения пожара, сохранность
электронного оборудования, диэлектрические свойства углекислого газа, что позволяет использовать эти огнетушители
даже в том случае, когда не удается обесточить электроустановку сразу. Для обнаружения начальной стадии
загорания и оповещения службу пожарной охраны используют системы автоматической пожарной сигнализации (АПС).
Кроме того, они могут самостоятельно приводить в действие установки пожаротушения, когда пожар еще не достиг
больших размеров. Системы АПС состоят из пожарных извещателей, линий связи и приемных пультов (станций).
Эффективность применения систем АПС определяется правильным выбором типа извещателей и мест их установки.
При выборе пожарных извещателей необходимо учитывать конкретные условия их эксплуатации: особенности помещения и
воздушной среды, наличие пожарных материалов, характер возможного горения, специфику технологического процесса и т.п.

Особое внимание уделяется пожарной безопасности, так как пожары в ВЦ сопряжены с опасностью для человеческой
жизни и большими материальными потерями.

В соответствии с "Типовыми правилами пожарной безопасности для промышленных предприятий" залы ЭВМ, помещения
для внешних запоминающих устройств, подготовки данных, сервисной аппаратуры, архивов, копировально множительного
оборудования и т.п. необходимо оборудовать дымовыми пожарными извещателями. В этих помещениях в начале пожара
при горении различных пластмассовых, изоляционных материалов и бумажных изделий выделяется значительное количество
дыма и мало теплоты. В других помещениях ВЦ, в том числе в машинных залах дизель генераторов и лифтов, трансформаторных
и кабельных каналах, воздуховодах допускается применение тепловых пожарных извещателей. Объекты ВЦ кроме АПС необходимо
оборудовать установками стационарного автоматического пожаротушения. Наиболее целесообразно применять в ВЦ установки
газового тушения пожара, действие которых основано на быстром заполнении помещения огнетушащим газовым веществом с
резким смижением содержания в воздухе кислорода.

\subsection{Оснащение помещения устойством для локального тушения пожаров}
Автоматические установки пожаротушения как правило проектируются с учетом ГОСТ 12.3.046, ГОСТ 15150, ПУЭ-98 и других нормативных
документов, действующих в этой области, а также строительных особенностей защищаемых зданий, помещений и сооружений, возможности
и условий применения огнетушащих веществ исходя из характера технологического процесса производства.
Необходимо добавить, что данный тип оборудования может выполнять и функции автоматической пожарной сигнализации.
С учетом пожарной опасности и физико-химических свойств производимых, хранимых и применяемых веществ и материалов необходимо
выбирать тип установки и огнетушащее вещество

\subsubsection{Установки пожаротушения водой, пеной низкой и средней кратности}
Установки водяного, пенного низкой кратности, а также водяного пожаротушения со смачивателем подразделяются на спринклерные и дренчерные.
При устройстве установок пожаротушения в помещениях, имеющих технологическое оборудование и площадки, горизонтально или наклонно
установленные вентиляционные короба с шириной или диаметром сечения свыше 0,75 м, расположенные на высоте не менее 0,7 м от плоскости
пола, если они препятствуют орошению защищаемой поверхности, следует дополнительно устанавливать спринклерные или дренчерные
оросители с побудительной системой под площадки, оборудование и короба.

Тип запорной арматуры (задвижки), применяемой в установках пожаротушения, должен обеспечивать визуальный контроль ее состояния
("закрыто", "открыто"). Допускается использование датчиков контроля положения запорной арматуры

\subsubsection{Спринклерные установки}

Спринклерные установки проектируются для помещений высотой не более 20 м, за исключением установок, предназначенных для защиты
конструктивных элементов покрытий зданий и сооружений.

В зависимости от температуры воздуха в помещениях спринклерные установки водяного и пенного пожаротушения могут быть:
\begin{itemize}
\item водозаполненными - для помещений с минимальной температурой воздуха 5${^\circ{C}}$ и выше;
\item воздушными - для неотапливаемых помещений зданий с минимальной температурой ниже 5${^\circ{C}}$.
\end{itemize}

Для одной секции спринклерной установки следует принимать не более 800 спринклерных оросителей всех типов.
При этом общая емкость трубопроводов каждой секции воздушных установок должна составлять не более 3,0 м$^3$.

При защите нескольких помещений, этажей здания одной спринклерной секцией для выдачи сигнала, уточняющего адрес загорания, а также
включения систем оповещения и дымоудаления допускается устанавливать на питающих трубопроводах сигнализаторы потока жидкости.

Для зданий с балочными перекрытиями (покрытиями) класса пожарной опасности К0 и К1 с выступающими частями высотой более 0,32 м, а
в остальных случаях - более 0,2 м, спринклерные оросители следует устанавливать между балками, ребрами плит и другими выступающими
элементами перекрытия (покрытия) с учетом обеспечения равномерности орошения пола.

В зданиях с односкатными и двухскатными покрытиями, имеющими уклон более 1/3, расстояние по горизонтали от спринклерных оросителей
до стен и от спринклерных оросителей до конька покрытия должно быть не более 1,5 м - при покрытиях с классом пожарной опасности
К0 и не более 0,8 м - в остальных случаях. В местах, где имеется опасность механического повреждения, спринклерные оросители
должны быть защищены специальными защитными решетками.

Спринклерные оросители водозаполненных установок необходимо устанавливать вертикально розетками вверх, вниз или горизонтально,
в воздушных установках - вертикально розетками вверх или горизонтально.

Спринклерные оросители установок следует устанавливать в помещениях или в оборудовании с максимальной температурой окружающего воздуха,
${^\circ{C}}$:
\begin{itemize}
\item до 41 - с температурой разрушения теплового замка 57-67${^\circ{C}}$;
\item до 50 - с температурой разрушения теплового замка 68-79${^\circ{C}}$;
\item от 51 до 70 - с температурой разрушения теплового замка 93${^\circ{C}}$;
\item от 71 до 100 - с температурой разрушения теплового замка 141${^\circ{C}}$;
\item от 101 до 140 - с температурой разрушения теплового замка 182${^\circ{C}}$;
\item 141 до 200 - с температурой разрушения теплового замка 240${^\circ{C}}$.
\end{itemize}

В пределах одного защищаемого помещения следует устанавливать спринклерные оросители с выпускным отверстием одного диаметра.

\subsubsection{Дренчерные установки}
Автоматическое включение дренчерных установок следует осуществлять по сигналам от одного из видов технических средств:
\begin{itemize}
\item побудительных систем;
\item установок пожарной сигнализации;
\item датчиков технологического оборудования.
\end{itemize}

Побудительный трубопровод дренчерных установок, заполненных водой или раствором пенообразователя, следует устанавливать на
высоте относительно клапана не более 1/4 постоянного напора (в метрах) в подводящем трубопроводе или в соответствии с
технической документацией на клапан, используемый в узле управления.

Для нескольких функционально связанных дренчерных завес допускается предусматривать один узел управления.
Включение дренчерных завес допускается осуществлять автоматически при срабатывании установки пожаротушения
дистанционно или вручную. Расстояние между оросителями дренчерных завес следует определять из расчета расхода
воды или раствора пенообразователя 1,0 л/с на 1 м ширины проема. Расстояние от теплового замка побудительной системы
до плоскости перекрытия (покрытия) должно быть от 0,08 до 0,4 м.

Заполнение помещения пеной при объемном пенном пожаротушении следует предусматривать до высоты, превышающей самую высокую
точку защищаемого оборудования не менее чем на 1 м.

При определении общего объема защищаемого помещения объем оборудования, находящегося в помещении, не следует вычитать
из защищаемого объема помещения.

\subsubsection{Установки пожаротушения высокократной пеной}
Установки пожаротушения высокократной пеной (далее по тексту раздела - установки) применяются для объемного и локально-объемного
тушения пожаров классов А2, В по ГОСТ 27331. Установки локально-объемного пожаротушения высокократной пеной применяются для
тушения пожаров отдельных агрегатов или оборудования в тех случаях, когда применение установок для защиты помещения в целом
технически невозможно или экономически нецелесообразно.

Классификация установок

По воздействию на защищаемые объекты установки подразделяются на:
- установки объемного пожаротушения;
- установки локального пожаротушения по объему.

По конструкции пеногенераторов установки подразделяются на:
\begin{itemize}
\item установки с генераторами, работающими с принудительной подачей воздуха (как правило, вентиляторного типа);
\item установки с генераторами эжекционного типа.
\end{itemize}

Проектирование

Установки должны обеспечивать заполнение защищаемого объема пеной до высоты, превышающей самую высокую точку оборудования не менее чем
на 1 м, в течение не более 10 мин. При эксплуатации рекомендуется использовать только специальные пенообразователи,
предназначенные для получения пены высокой кратности. Производительность и количество раствора пенообразователя определяются
исходя из расчетного объема защищаемых помещений. При применении для локального пожаротушения по объему защищаемые агрегаты
или оборудование ограждаются металлической сеткой с размером ячейки не более 5 мм. Высота ограждающей конструкции должна быть
на 1 м больше высоты защищаемого агрегата или оборудования и находиться от него на расстоянии не менее 0,5 м. Установки должны
быть снабжены фильтрующими элементами, установленными на питающих трубопроводах перед распылителями, размер фильтрующей ячейки
должен быть меньше минимального размера канала истечения распылителя. При расположении генераторов пены в местах их возможного
механического повреждения должна быть предусмотрена их защита. Кроме расчетного количества должен быть 100\%-ный резерв
пенообразователя.

\subsubsection{Установки пожаротушения тонкораспыленной водой}
Установки пожаротушения тонкораспыленной водой (далее по тексту раздела - установки) применяются для поверхностного и локального
по поверхности тушения очагов пожара классов А, В. Исполнение должно соответствовать требованиям НПБ 80-99.

При использовании воды с добавками, выпадающими в осадок или образующими раздел фаз при длительном хранении, в установках
должны быть предусмотрены устройства для их перемешивания. Для модульных установок в качестве газа-вытеснителя применяются
воздух, инертные газы, СО$_2$, N$_2$. Сжиженные газы, применяемые в качестве вытеснителей огнетушащего вещества, не должны
ухудшать параметры работы установки.

В установках для вытеснения огнетушащего вещества допускается применение газогенерирующих элементов, прошедших промышленные
испытания и рекомендованных к применению в пожарной технике. Конструкция газогенерирующего элемента должна исключать
возможность попадания в огнетушащее вещество каких-либо его фрагментов.

Запрещается применение газогенерирующих элементов в качестве вытеснителей огнетушащего вещества при защите культурных ценностей.
Выходные отверстия насадков (распылителей) должны быть защищены от загрязняющих факторов внешней среды. Защитные приспособления
(декоративные корпуса, колпачки) не должны ухудшать параметров работы установок.

Если на одном объекте применяются модульные установки разного типоразмера, то запас модулей должен обеспечивать восстановление
работоспособности установок, защищающих помещения наибольшего объема модулями каждого типоразмера. Нормативные параметры подачи
тонкораспыленной воды и методика расчета установок принимаются по техническим условиям, разрабатываемыми для каждого
конкретного объекта и согласованными с ГУГПС МВД России.

\subsubsection{Установки газового пожаротушения}
Установки газового пожаротушения применяются для ликвидации пожаров классов А, В, С по ГОСТ 27331 и электрооборудования (электроустановок с напряжением не выше указанного в ТД на используемые газовые огнетушащие вещества (ГОТВ)).
При этом установки не должны применяться для тушении пожаров:
\begin{itemize}
\item волокнистых, сыпучих, пористых и других горючих материалов, склонных к самовозгоранию и/или тлению внутри объема вещества
(древесные опилки, хлопок, травяная мука и др.);
\item химических веществ и их смесей, полимерных материалов, склонных к тлению и горению без доступа воздуха;
\item гидридов металлов и пирофорных веществ;
\item порошков металлов (натрий, калий, магний, титан и др).
\end{itemize}

Для установок азотного и аргонного пожаротушения параметр негерметичности не должен превышать 0,001 м.

Классификация и состав установок

Установки подразделяются:
\begin{itemize}
\item по способу тушения: объемного тушения, локального по объему;
\item по способу хранения газового огнетушащего вещества: централизованные, модульные;
\item по способу включения от пускового импульса: с электрическим, пневматическим, механическим пуском или их комбинацией.
\end{itemize}

Технологическая часть установки (типовой вариант) в зависимости от способа хранения газового огнетушащего вещества и конструктивного
исполнения содержит:
\begin{description}
\item[a)] модульная установка:
	\begin{itemize}
	\item модули газового пожаротушения (далее по тексту - модули);
	\item распределительные трубопроводы;
	\item насадки.
	\end{itemize}
\item[б)] централизованная установка:
	\begin{itemize}
	\item батареи газового пожаротушения (далее по тексту - батареи), модули и/или изотермические резервуары, размещенные в помещении станции пожаротушения;
	\item коллектор в станции пожаротушения и установленные на нем распределительные устройства;
	\item магистральный и распределительный трубопроводы;
	\item насадки.
	\end{itemize}
\end{description}
Кроме того, в состав технологической части установки может входить побудительная система.

Проектирование-общие требования

Установки должны соответствовать требованиям ГОСТ Р 50969. Исполнение оборудования, входящего в состав установки, должно
соответствовать требованиям действующей нормативной документации. При составлении проекта технологической части установки
производят расчеты по определению:
\begin{itemize}
\item массы ГОТВ в установке пожаротушения;
\item диаметра трубопроводов установки, типа и количества насадков, времени подачи ГОТВ (гидравлический расчет);
\item площади проема для сброса избыточного давления в защищаемом помещении при подаче газового огнетушащего вещества.
\end{itemize}

\subsubsection{Установки объемного пожаротушения}

Исходными данными для расчета и проектирования установки являются:
\begin{itemize}
\item перечень помещений и наличие пространств фальшполов и подвесных потолков, подлежащих защите установкой пожаротушения;
\item количество помещений (направлений), подлежащих одновременной защите установкой пожаротушения;
\item геометрические параметры помещения (конфигурация помещения, длина, ширина и высота ограждающих конструкций);
\item конструкция перекрытий и расположение инженерных коммуникаций;
\item площадь постоянно открытых проемов в ограждающих конструкциях и их расположение;
\item предельно допустимое давление в защищаемом помещении;
\item диапазон температуры, давления и влажности в защищаемом помещении и в помещении, в котором размещаются
составные части установки;
\item перечень и показатели пожарной опасности веществ и материалов, находящихся в помещении, и соответствующий им класс пожара
по ГОСТ 27331;
\item тип, величина и схема распределения пожарной нагрузки;
\item наличие и характеристика систем вентиляции, кондиционирования воздуха, воздушного отопления;
\item характеристика технологического оборудования;
\item категория помещений по НПБ 105-95 и классы зон по ПУЭ-98;
\item наличие людей и пути их эвакуации.
\end{itemize}

Расчетное количество (масса) ГОТВ в установке должно быть достаточным для обеспечения его нормативной огнетушащей концентрации в
любом защищаемом помещении или группе помещений, защищаемых одновременно. Централизованные установки, кроме расчетного количества
ГОТВ, должны иметь его 100\%-ный резерв. Допускается совместное хранение расчетного количества и резерва ГОТВ в изотермическом
резервуаре при условии оборудования последнего запорно-пусковым устройством с реверсивным приводом и техническими средствами
его управления. Модульные установки, кроме расчетного количества ГОТВ, должны иметь его 100\%-ный запас. При наличии на объекте
нескольких модульных установок запас предусматривается в объеме, достаточном для восстановления работоспособности установки,
сработавшей в любом из защищаемых помещений объекта.При необходимости испытаний установки запас ГОТВ на проведение указанных
испытаний принимается из условия защиты помещения наименьшего объема, если нет других требований.

Установка должна обеспечивать задержку выпуска газового огнетушащего вещества в защищаемое помещение при автоматическом и
дистанционном пуске на время, необходимое для эвакуации из помещения людей, отключение вентиляции (кондиционирования и т. п.),
закрытие заслонок (противопожарных клапанов и т. д.), но не менее 10 с от момента включения в помещении устройств оповещения об
эвакуации. Установка должна обеспечивать инерционность (время срабатывания без учета времени задержки выпуска ГОТВ) не более 15 с.

Установка должна обеспечивать подачу не менее 95\% массы газового огнетушащего вещества, требуемой для создания нормативной
огнетушащей концентрации в защищаемом помещении, за временной интервал, не превышающий: 10 с для модульных установок, в которых
в качестве ГОТВ применяются сжиженные газы (кроме двуокиси углерода); 15 с для централизованных установок, в которых в качестве
ГОТВ применяются сжиженные газы (кроме двуокиси углерода);60 с для модульных и централизованных установок,
в которых в качестве ГОТВ применяются двуокись углерода или сжатые газы.

Номинальное значение временного интервала определяется при хранении сосуда с ГОТВ при температуре 20$^\circ{C}$.

В установках применяются следующие сосуды: модули газового пожаротушения; батареи газового пожаротушения; изотермические резервуары.

В централизованных установках сосуды следует размещать в станциях пожаротушения. В модульных установках модули могут располагаться как
в самом защищаемом помещении, так и за его пределами, в непосредственной близости от него. Расстояние от сосудов до источников
тепла (приборов отопления и т. п.) должно составлять не менее 1 м. Распределительные устройства следует размещать в помещении
станции пожаротушения. Размещение технологического оборудования централизованных и модульных установок должно обеспечивать
возможность их обслуживания. Сосуды следует размещать возможно ближе к защищаемым помещениям. При этом сосуды не следует
располагать в местах, где они могут быть подвергнуты опасному воздействию факторов пожара (взрыва), механическому, химическому
или иному повреждению, прямому воздействию солнечных лучей.
Сосуды в составе установки должны быть надежно закреплены в соответствии с эксплуатационными документами на сосуды.
В установках, где в качестве ГОТВ используются сжиженные газы, следует предусмотреть технические средства, обеспечивающие
контроль массы ГОТВ в соответствии с ГОСТ Р 50969 и ТД на модули или изотермические резервуары.

При этом модули, содержащие ГОТВ-сжиженные газы без газа-вытеснителя, должны быть оборудованы устройствами контроля его массы в
соответствии с НПБ 54-96. При использовании в качестве ГОТВ сжатого газа, а также газа-вытеснителя, сосуды обеспечиваются
устройствами контроля давления.
Размещение термочувствительных элементов побудительных систем в защищаемых помещениях производится в соответствии с
требованиями, приведенными в разделе "Установки пожаротушения водой, пеной низкой и средней кратности".

Диаметр условного прохода побудительных трубопроводов следует принимать равным 15 мм. Устройства дистанционного пуска
установки должны располагаться на высоте не более 1,7 м.
Остальные требования к устройствам дистанционного пуска должны соответствовать требованиям к аналогичным устройствам АУГП,
изложенным в разделах 11-14 и действующей нормативной документации.

Выбор типа насадков определяется их техническими характеристиками для конкретного ГОТВ.
Насадки должны размещаться в защищаемом помещении с учетом его геометрии и обеспечивать распределение ГОТВ по всему
объему помещения с концентрацией не ниже нормативной.
Насадки, установленные на трубопроводной разводке для подачи ГОТВ, плотность которых при нормальных условиях больше плотности
воздуха, должны быть расположены на расстоянии не более 0,5 м от перекрытия (потолка, подвесного потолка, фальшпотолка)
защищаемого помещения.

Прочность насадков должна обеспечиваться при давлении 1,25 Pраб. Поверхность выпускных отверстий насадков должна быть выполнена
из коррозионно-стойкого материала. Выпускные отверстия насадков должны быть ориентированы таким образом, чтобы струи ГОТВ не
были непосредственно направлены в постоянно открытые проемы защищаемого помещения. При расположении насадков в местах их
возможного механического повреждения или засорения они должны быть защищены.

Помещения станций пожаротушения должны быть отделены от других помещений противопожарными перегородками 1-го типа и перекрытиями
3-го типа. Помещения станции нельзя располагать под и над помещениями категорий А и Б. В помещениях станций пожаротушения
должна быть температура от 5 до 35 oС, относительная влажность воздуха не более 80\% при 25$^\circ{C}$, освещенность - не
менее 100 лк при люминесцентных лампах или не менее 75 лк при лампах накаливания. Аварийное освещение должно соответствовать
требованиям СНиП 23.05-95.

Помещения станций должны быть оборудованы приточно-вытяжной вентиляцией с не менее чем двукратным воздухообменом, а также
телефонной связью с помещением дежурного персонала, ведущим круглосуточное дежурство.

У входа в помещение станции должно быть установлено световое табло "Станция пожаротушения". Входная дверь должна иметь запорное
устройство, исключающее несанкционированный доступ в помещение станции пожаротушения.

Размещение приборов и оборудования в станции пожаротушения должно обеспечивать возможность их обслуживания.

Централизованные установки должны быть оснащены устройствами местного пуска. Местный пуск модульных установок, модули которых
размещены в защищаемом помещении, должен быть исключен. При наличии пусковых элементов на модулях они должны быть блокированы.
Пусковые элементы устройств местного пуска должны располагаться на высоте не более 1,7 м от пола.

\subsubsection{Установки порошкового пожаротушения модульного типа}

Установки порошкового пожаротушения применяются для локализации и ликвидации пожаров классов А, В, С и электрооборудования
в соответствии с данными на огнетушащий порошковый состав, которым они заряжены.

При защите помещений, относящихся к взрывопожароопасной категории (категории А и Б по НПБ 105-95 и взрывоопасные зоны по ПУЭ),
оборудование входящее в состав установки, при его размещении в защищаемом помещении, должно иметь взрывобезопасное исполнение.
Установки могут применяться для локализации или тушения пожара на защищаемой площади, локального тушения на части площади или
объема, тушения всего защищаемого объема.

В помещениях с массовым пребыванием людей (театры, торговые комплексы и др.) установки должны выполняться в соответствии с требованиями
ГОСТ 12.3.046 и требованиями раздела 11 (п.п. 11.11 - 11.16) настоящего документа.

Для защиты помещений объемом не более 100м$^3$, где не предусмотрено постоянное пребывание людей и посещение которых производится
периодически (по мере производственной необходимости), в которых горючая загрузка не превышает 50 кг/м$^2$, скорости воздушных
потоков в зоне тушения не превышают 1,5 м/с, а также для защиты электрошкафов, кабельных сооружений и др., допускается,
при отдельном выполнении автоматической пожарной сигнализации, применение установок, осуществляющих только функции обнаружения
и тушения пожара.

Проектирование

В проектной документации на установку должны быть отражены параметры установки в соответствии с ГОСТ Р 51091 и правила ее эксплуатации.
В зависимости от конструкции модуля порошкового пожаротушения установки могут быть с распределительным трубопроводом или без него.
По способу хранения вытесняющего газа в модуле (емкости) установки подразделяются на: закачные, с газогенерирующим элементом, с
баллоном сжатого или сжиженного газа.

При расчете объема защищаемого помещения, в случае, когда оборудование и строительные конструкции выполнены из негорючих материалов,
допускается вычитать их объем из расчетного объема помещения.

Локальная защита отдельных производственных зон, участков, агрегатов и оборудования производится в помещениях со скоростями
воздушных потоков не более 1,5 м/с, или с параметрами указанными в технической документации (ТД) на модуль порошкового пожаротушения.
За расчетную зону локального пожаротушения принимается увеличенный на 10\% размер защищаемой площади, увеличенный на 15\%
размер защищаемого объема.

Тушение всего защищаемого объема помещения допускается предусматривать в помещениях со степенью негерметичности до 1,5\%.
В помещениях объемом свыше 400м$^3$ , как правило, применяются способы пожаротушения - локальный по площади или объему, или
по всей площади. Максимальная длина распределительных трубопроводов и требования к ним регламентируются ТД на модули порошкового
тушения, трубопроводы следует выполнять из стальных труб.

Соединения трубопроводов в установках пожаротушения должны быть сварными, фланцевыми или резьбовыми. Трубопроводы и их соединения
в установках пожаротушения должны обеспечивать герметичность при испытательном давлении, равном Р. Трубопроводы и их соединения
в установках пожаротушения должны обеспечивать прочность при испытательном давлении, равном 1,25 Р. Конструкции,
используемые для установки модулей или трубопроводов с насадками-распылителями, должны выдерживать воздействие нагрузки,
равной пятикратному весу устанавливаемых элементов, и обеспечивать их сохранность и защиту от случайных повреждений.

Модули порошкового пожаротушения рекомендуется размещать с учетом диапазона температур эксплуатации.
Расчет количества модулей, необходимого для пожаротушения, должен осуществляться из условия обеспечения равномерного заполнения
огнетушащим порошком защищаемого объема или равномерного орошения площади с учетом диграмм распыла ( приведенных в ТД на модуль).

При использовании установки (при обосновании в проекте) может применяться резервирование. При этом общее количество модулей
удваивается по сравнению с расчетным и производится двухступенчатый запуск модулей. Для включения второй ступени допускается
применение дистанционного управления.

\subsection{Экологическая оценка проектируемой компьютерной техники}
Внедрение в промышленность новых, более эффективных промышленных процессов, резкое повышение продуктивности и расширение
масштабов производства потребовали увеличения затрат материальных и энергетических ресурсов, что, в свою очередь,
привело к росту отрицательного воздействия на окружающую среду. Основными проблемами по решению задач защиты окружающей
среды являются: совершенствование технологических процессов и разработка нового оборудования с меньшим уровнем выброса
примесей и отходов в окружающую среду, также необходимо уменьшить влияние таких факторов как шум при работе, излучение
высокочастотных электромагнитных полей, сильный разогрев и т.п.

При производстве плат должны предусматриваться эффективные средства
защиты окружающей среды от возможного загрязнения. В технологии производства плат информационно-управляющих систем
компьютерной используются процессы, отрицательного воздействия на окружающую среду, такие как термическая,
гальваническая обработка, пайка и окраска.

Гальванические работы сопряжены с использованием больших объёмов воды для приготовления растворов электролитов и
промывочных операций. Поэтому сточные воды в этих случаях значительно загрязнены ядовитыми химическими веществами.
Кроме того, воздух, удаляемый от технологического гальванического оборудования, содержит большое количество вредных
веществ в различных агрегатных состояниях: капель но-жид ком, паро- и газообразном.Технологические процессы сварки и
пайки сопровождаются выделением пыли и токсичных газов, а сточные воды могут загрязняться механическими примесями,
кислотами. Процесс получения функционально завершённого изделия заканчивается сборочными операциями. Отрицательное
воздействие на окружающую среду процессов сборки менее ощутимо. Однако и в этих случаях при проведении
санитарно-гигиенической обработки производственных помещений в сточные воды могут попадать различные нежелательные примеси.

В настоящее время широко используются пассивные методы защиты, суть которых сводится к ограничению количества
загрязняющих окружающую среду выбросов, т.е. улавливанию пылегазовыделений, выбрасываемых в атмосферу, очистка
сточных вод от примесей и т.п. При производстве модулей должны использоваться пассивные фильтры, которые основаны
на способности пористых материалов задерживать частицы примесей при движении дисперсных сред. Частицы примесей оседают
на входной части фильтроэлемента, помещённого в корпус. Осаждение частиц происходит в результате совокупного действия
эффекта касания, диффузионных, инерционных гравитационных процессов. Для очистки воздуха от туманов кислот, щелочей,
масел и других жидкостей используются волоконные фильтры, принцип действия которых основан на осаждении капель на
поверхности материалов и последующего отекания жидкостей под воздействием сил тяжести.

При загрязнении сточных вод маслосо держащим и примесями, помимо отстаивания и фильтрования, применяется также
процесс флотации. Очистка вод флотацией заключается в интенсификации процесса маслопродуктов при их частиц
пузырьками воздуха, попадающего в сточную воду. Таким образом, наиболее перспективной формой защиты окружающей
среды от вредного воздействия является «безотходная» технология и комплекс природоохранных мероприятий в
технологических процессах от обработки сырья до использования готовой продукции.

\newpage

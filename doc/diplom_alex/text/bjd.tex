\section{БЖД}

\subsection{Анализ пожарной опасности в помещении небольших размеров, где установлена вычислительная техника}

Пожары в ВЦ представляют особую опасность, так как сопряжены с большими материальными потерями.
Характерная особенность ВЦ - небольшие площади помещений. Как известно пожар может возникнуть при взаимодействии
горючих веществ, окисления и источников зажигания. В помещениях ВЦ присутствуют все три основные фактора,
необходимые для возникновения пожара. Горючими компонентами на ВЦ являются: строительные материалы для
акустической и эстетической отделки помещений, перегородки, двери, полы, перфокарты и перфоленты, изоляция кабелей и др.

Противопожарная защита - это комплекс организационных и технических мероприятий, направленных на обеспечение безопасности людей, на предотвращение пожара, ограничение его распространения, а также на создание условий для успешного тушения пожара.

Источниками зажигания в ВЦ могут быть электронные схемы от ЭВМ, приборы, применяемые для технического обслуживания, устройства электропитания, кондиционирования воздуха, где в результате различных нарушений образуются перегретые элементы, электрические искры и дуги, способные вызвать загорания горючих материалов.

В современных ЭВМ очень высокая плотность размещения элементов электронных схем. В непосредственной близости
друг от друга располагаются соединительные провода, кабели. При протекании по ним электрического тока выделяется
значительное количество теплоты. При этом возможно оплавление изоляции. Для отвода избыточной теплоты от ЭВМ служа
системы вентиляции и кондиционирования воздуха. При постоянном действии эти системы представляют собой дополнительную
пожарную опасность. Энергоснабжение ВЦ осуществляется от трансформаторной станции и двигатель-генераторных агрегатов.
На трасформаторных подстанциях особую опасность представляют трансформаторы с масляным охлаждением. В связи с этим
предпочтение следует отдавать сухим транформатором. Пожарная опасность двигатель-генераторных агрегатов обусловленна
возможностью коротких замыканий, перегрузки, электрического искрения. Для безопасной работы необходим правильный
расчет и выбор аппаратов защиты. При поведении обслуживающих, ремонтных и профилактических работ используются
различные смазочные вещества, легковоспламеняющиеся жидкости, прокладываются временные электропроводники, ведут пайку
и чистку отдельных узлов. Возникает дополнительная пожарная опасность, требующая дополнительных мер пожарной защиты.
В частности, при работе с паяльником следует использовать несгораемую подставку с несложными приспособлениями
для уменьшения потребляемой мощности в нерабочем состоянии.

Для большинства помещений ВЦ установлена категория пожарной опасности. Одной из наиболее важных задач пожарной защиты
является защита строительных помещений от разрушений и обеспечение их достаточной прочности в условиях воздействия
высоких температур при пожаре. Учитывая высокую стоимость электронного оборудования ВЦ, а также категорию его
пожарной опасности, здания для ВЦ и части здания другого назначения, в которых предусмотрено размещение ЭВМ должны
быть 1 и 2 степени огнестойкости. Для изготовления строительных конструкций используются, как правило, кирпич,
железобетон, стекло, металл и другие негорючие материалы. Применение дерева должно быть ограниченно, а в
случае использования необходимо пропитывать его огнезащитными составами. В ВЦ противопожарные преграды в виде
перегородок из несгораемых материалов устанавливают между машинными залами.

К средствам тушения пожара, предназначенных для локализации небольших загораний, относятся пожарные стволы,
внутренние пожарные водопроводы, огнетушители, сухой песок, асбестовые одеяла и т. п. В зданиях ВЦ пожарные
краны устанавливаются в коридорах, на площадках лестничных клеток и входов. Вода используется для тушения пожаров
в помещениях программистов, библиотеках, вспомогательных и служебных помещениях. Применение воды в машинных залах
ЭВМ, хранилищах носителей информации, помещениях контрольно измерительных приборов ввиду опасности повреждения или
полного выхода из строя дорогостоящего оборудования возможно в исключительных случаях, когда пожар принимает
угрожающе крупные размеры. При этом количество воды должно быть минимальным, а устройства ЭВМ необходимо защитить
от попадания воды, накрывая их бризентом или полотном.

Для тушения пожаров на начальных стадиях широко применяются огнетушители. По виду используемого огнетушащего
вешества огнетушители подразделяются на следующие основные группы. Пенные огнетушители, применяются для тушения
горящих жидкостей, различных материалов, конструктивных элементов и оборудования, кроме электрооборудования,
находящегося под напряжением. Газовые огнетушители применяются для тушения жидких и твердых веществ, а также
электроустановок, находящихся под напряжением. В производственных помещениях ВЦ применяются главным образом
углекислотные огнетушители, достоинством которых является высокая эффективность тушения пожара, сохранность
электронного оборудования, диэлектрические свойства углекислого газа, что позволяет использовать эти огнетушители
даже в том случае, когда не удается обесточить электроустановку сразу. Для обнаружения начальной стадии
загорания и оповещения службу пожарной охраны используют системы автоматической пожарной сигнализации (АПС).
Кроме того, они могут самостоятельно приводить в действие установки пожаротушения, когда пожар еще не достиг
больших размеров. Системы АПС состоят из пожарных извещателей, линий связи и приемных пультов (станций).
Эффективность применения систем АПС определяется правильным выбором типа извещателей и мест их установки.
При выборе пожарных извещателей необходимо учитывать конкретные условия их эксплуатации: особенности помещения и
воздушной среды, наличие пожарных материалов, характер возможного горения, специфику технологического процесса и т.п.

Особое внимание уделяется пожарной безопасности, так как пожары в ВЦ сопряжены с опасностью для человеческой
жизни и большими материальными потерями.

\subsection{Оснащение помещения устойством для локального тушения пожаров}

В соответствии с "Типовыми правилами пожарной безопасности для промышленных предприятий" залы ЭВМ, помещения
для внешних запоминающих устройств, подготовки данных, сервисной аппаратуры, архивов, копировально множительного
оборудования и т.п. необходимо оборудовать дымовыми пожарными извещателями. В этих помещениях в начале пожара
при горении различных пластмассовых, изоляционных материалов и бумажных изделий выделяется значительное количество
дыма и мало теплоты. В других помещениях ВЦ, в том числе в машинных залах дизель генераторов и лифтов, трансформаторных
и кабельных каналах, воздуховодах допускается применение тепловых пожарных извещателей. Объекты ВЦ кроме АПС необходимо
оборудовать установками стационарного автоматического пожаротушения. Наиболее целесообразно применять в ВЦ установки
газового тушения пожара, действие которых основано на быстром заполнении помещения огнетушащим газовым веществом с
резким смижением содержания в воздухе кислорода.

\subsection{Экологическая оценка проектируемой компьютерной техники}
Внедрение в промышленность новых, более эффективных промышленных процессов, резкое повышение продуктивности и расширение
масштабов производства потребовали увеличения затрат материальных и энергетических ресурсов, что, в свою очередь,
привело к росту отрицательного воздействия на окружающую среду. Основными проблемами по решению задач защиты окружающей
среды являются: совершенствование технологических процессов и разработка нового оборудования с меньшим уровнем выброса
примесей и отходов в окружающую среду, также необходимо уменьшить влияние таких факторов как шум при работе, излучение
высокочастотных электромагнитных полей, сильный разогрев и т.п. Технология, которая используется при проектировании
устройства - безвредна, так как были учтены все ограничения ведущие к снижению и исключению влияния устройства на
окружающую среду. Например, при пайке электронных устройств, применяется припой, содержащий свинец, который, испаряясь,
загрязняет воздух рабочей зоны.

При производстве плат входящих в состав модернизированной систему с ЧПУ должны предусматриваться эффективные средства
защиты окружающей среды от возможного загрязнения. В технологии производства плат информационно-управляющей
компьютерной системы используются процессы, отрицательного воздействия на окружающую среду, такие как термическая,
гальваническая обработка, пайка и окраска.

Гальванические работы сопряжены с использованием больших объёмов воды для приготовления растворов электролитов и
промывочных операций. Поэтому сточные воды в этих случаях значительно загрязнены ядовитыми химическими веществами.
Кроме того, воздух, удаляемый от технологического гальванического оборудования, содержит большое количество вредных
веществ в различных агрегатных состояниях: капель но-жид ком, паро- и газообразном.Технологические процессы сварки и
пайки сопровождаются выделением пыли и токсичных газов, а сточные воды могут загрязняться механическими примесями,
кислотами. Процесс получения функционально завершённого изделия заканчивается сборочными операциями. Отрицательное
воздействие на окружающую среду процессов сборки менее ощутимо. Однако и в этих случаях при проведении
санитарно-гигиенической обработки производственных помещений в сточные воды могут попадать различные нежелательные примеси.

В настоящее время широко используются пассивные методы защиты, суть которых сводится к ограничению количества
загрязняющих окружающую среду выбросов, т.е. улавливанию пылегазовыделений, выбрасываемых в атмосферу, очистка
сточных вод от примесей и т.п. При производстве модулей должны использоваться пассивные фильтры, которые основаны
на способности пористых материалов задерживать частицы примесей при движении дисперсных сред. Частицы примесей оседают
на входной части фильтроэлемента, помещённого в корпус. Осаждение частиц происходит в результате совокупного действия
эффекта касания, диффузионных, инерционных гравитационных процессов. Для очистки воздуха от туманов кислот, щелочей,
масел и других жидкостей используются волоконные фильтры, принцип действия которых основан на осаждении капель на
поверхности материалов и последующего отекания жидкостей под воздействием сил тяжести.

При загрязнении сточных вод маслосо держащим и примесями, помимо отстаивания и фильтрования, применяется также
процесс флотации. Очистка вод флотацией заключается в интенсификации процесса маслопродуктов при их частиц
пузырьками воздуха, попадающего в сточную воду. Таким образом, наиболее перспективной формой защиты окружающей
среды от вредного воздействия является «безотходная» технология и комплекс природоохранных мероприятий в
технологических процессах от обработки сырья до использования готовой продукции.

\newpage

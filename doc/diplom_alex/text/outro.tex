\addcontentsline{toc}{section}{ЗАКЛЮЧЕНИЕ}
\section*{ЗАКЛЮЧЕНИЕ}
Для разработки программно-аппаратной платформы был использован язык описания аппаратного обеспечения VHDL, программная часть реализована
под платформу Linux на языке C, модуль публикации дампов в публичные источники реализован на скрипт-языке Bash. Разработанный комплекс
является частью проекта по созданию программной платформы GPS приемника. 

Результаты тестовой эксплуатации разработанных аппаратно-программных средств позволяют сделать следующие выводы:
\begin{itemize}
\item выбранные инструментальные средства разработки и технологии  (ПЛИС, бесплатные САПР на основе языка VHDL, плаьформа LINUX) позволили
	полностью реализовать необходимую функциональность с минимальными затратами;
\item разработанный программно-аппаратный комплекс успешно применяется при проведении лабораторных работ на кафедре ИТ-6 МГУПИ в
	рамках курса "Современные информационные технологии";
\item дальнейшее развитие комплекса может позволить прием и обработку сигналов ГЛОНАСС, КОМПАС, ГАЛИЛЕО и др.
\end{itemize}

%В разделе "Безопасность жизнедеятельности" был проведен анализ пожарной опасности в помещении небольших размеров,
%где установлена ВТ. Разработаны мероприятия по оснащению помещения устройством для локального тушения пожаров. Дана
%экологическая оценка проектируемой компьютерной техники.

\newpage

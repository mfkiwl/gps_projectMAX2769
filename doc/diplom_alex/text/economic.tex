\section{ОРГАНИЗАЦИОННО-ЭКОНОМИЧЕСКИЙ РАЗДЕЛ}
\subsection{Планирование разработки программных средств с построением графика}
Целью дипломного проекта является разработка программно-аппаратного комплекса (ПК) захвата GNSS сигналов.
В данном разделе определяется трудоёмкость и затраты на создание ПК, а так же производится расчёт основных технико-экономических
показателей проекта.

\subsubsection*{Определение трудоемкости и продолжительности работ по созданию ПК}
Процесс разработки включает: обзор и анализ программных средств схожей тематики, анализ и выбор программных продуктов для
создания программы; отладка; испытание. В свою очередь каждый из этих этапов можно подразделить на отдельные под этапы.
Согласно ГОСТ 23501.1-79 регламентируются следующие стадии проведения исследования:

\begin{itemize}
	\item техническое задание – ТЗ (ГОСТ 23501.2-79);
	\item эскизный проект – ЭП (ГОСТ 23501.5-80);
	\item технический проект – ТП (ГОСТ 23501.6-80);
	\item рабочий проект – РП (ГОСТ 23501.11-81);
	\item внедрение – ВП (ГОСТ 23501.15-81).
\end{itemize}

Планирование стадий и содержания работ осуществляется в соответствии с \cite{bibl51}. На всех стадиях проведения исследования
выполняются следующие виды работ, перечень которых показан в таблице ~\ref{tab:eco1}.

% http://users.sdsc.edu/~ssmallen/latex/longtable.html
\begin{center}
\begin{longtable}{|l|l|}
\caption{Состав работ и стадии разработки ПК} \label{tab:eco1} \\ \hline
\multicolumn{1}{|c|}{\textbf{Стадии разработки}}    &   \multicolumn{1}{c|}{\textbf{Перечень работ}} \\ \hline
\multicolumn{1}{|c|}{\textbf{1}}    &   \multicolumn{1}{|c|}{\textbf{2}} \\ \hline
\endfirsthead

%\multicolumn{2}{c} %
%{{\bfseries \tablename \thetable{} -- continued from previous page}} \\
\multicolumn{2}{|l|}{{Продолжение таблицы ~\ref{tab:eco1}}} \\ %\hline
\hline \multicolumn{1}{|c|}{\textbf{1}} &
%\multicolumn{1}{c|}{\textbf{Triple chosen}} &
\multicolumn{1}{c|}{\textbf{2}} \\ \hline 
\endhead

%\multicolumn{2}{|r|}{{}} \\ %\hline
%\hline \multicolumn{2}{|r|}{{Continued on next page}} \\ %\hline
\endfoot

\hline
\endlastfoot
		Техническое задание & \begin{parbox}{5in} {
						\begin{itemize}
							\item постановка задачи;
							\item подбор литературы;
							\item сбор исходных данных;
							\item определение требований к системе;
							\item определение стадий, этапов и сроков разработки ПК;
						\end{itemize} }  
					\end{parbox} \\
	\hline
		Эскизный проект & \begin{parbox}{5in} {
					\begin{itemize} 
						\item анализ программных средств схожей тематики;
						\item разработка общей структуры ПК;
						\item разработка структуры программы по подсистемам;
						\item документирование;
					\end{itemize} }
				\end{parbox} \\
	\hline
		Технический проект & \begin{parbox}{5in} {
					\begin{itemize} 
						\item определение требований к ПК;
						\item выбор инструментальных средств;
						\item определение свойств и требований к аппаратному обеспечению;
					\end{itemize} } 
				\end{parbox} \\
	\hline
		Рабочий проект & \begin{parbox}{5in} {
					\begin{itemize} 
						\item программирование;
						\item тестирование и отладка ПК;
						\item разработка программной документации;
						\item согласование и утверждение программы и методики испытаний;
					\end{itemize} }
				\end{parbox} \\
	\hline
		Внедрение & \begin{parbox}{5in} {
					\begin{itemize}
						\item опытная эксплуатация;
						\itemнализ данных полученных в результате эксплуатации;
						\item корректировка технической документации по результатам испытаний;
					\end{itemize} }
				\end{parbox} \\
\end{longtable}
\end{center}

Трудоемкость разработки ПК определяется по сумме трудоемкости этапов и видов работ, оцениваемых экспертным путем в
человеко-днях, и носит вероятностный характер, так как зависит от множества трудно учитываемых факторов.

Трудоемкость каждого вида работ определяется в соответствии с методическими указаниями \cite{bibl53} по формуле:

\begin{equation}
t_i = \frac{3\cdot{t_{min}} + 2\cdot{t_{max}}}{5},
\label{eq:eco1}
\end{equation}
где:	$t_{min}$ - минимально возможная трудоемкость выполнения отдельного вида работ в человеко-днях; \\
	$t_{max}$ - максимально возможная трудоемкость выполнения отдельного вида работ в человеко-днях.

Продолжительность каждого вида работ в календарных днях ${(T_i)}$ определяется по формуле \cite{bibl53}:

\begin{equation}
T_i = \frac{t_i}{\mbox{Ч}_i}\cdot{K_{\mbox{вых}}},
\label{eq:eco2}
\end{equation}
где:	${t_i}$ - трудоемкость работы, человек-дней; \\
	${\mbox{Ч}_i}$ - численность исполнителей, человек; \\
	${K_{\mbox{вых}}}$ - коэффициент, учитывающий выходные и праздничные дни: \\

\begin{equation}
T_i = \frac{K_{\mbox{кап}}}{K_{\mbox{раб}}},
\label{eq:eco3}
\end{equation}
где:	${K_{\mbox{кап}}}$ - число календарных дней;
	${K_{\mbox{раб}}}$ - рабочие дни.

Для расчёта принимается среднее значение ${K_{\mbox{вых}} = 1.45}$. 

Полный список видов и этапов работ по созданию ПС, экспертные оценки и расчетные величины их трудоемкости,
а также продолжительность каждого вида работ, рассчитанные по формулам (\ref{eq:eco1}) и (\ref{eq:eco2}) представлены
в таблице ~\ref{tab:eco2}.

\begin{center}
\begin{longtable}{|c|c|c|c|c|c|c|}
\caption{Расчёт трудоёмкости и продолжительности работ по созданию ПК} \label{tab:eco2} \\ \hline
\multicolumn{1}{|c|}{\textbf{№}} & \multicolumn{1}{p{3.5cm}|}{\textbf{Наименование стадий и  работ}} & 
\multicolumn{3}{p{3.5cm}|}{\textbf{Трудоемкость, человеко-дни}} &   \multicolumn{1}{p{3cm}|}{\textbf{Кол-во исполнителей, чел.}} &
\multicolumn{1}{p{3.8cm}|}{\textbf{Продолжитель-ность, календарные дни}} \\
\cline{3-7}

\multicolumn{1}{|c|}{} &   \multicolumn{1}{c|}{} & 
\multicolumn{1}{|c|}{${t_{min}}$} & \multicolumn{1}{c|}{${t_{max}}$} & 
\multicolumn{1}{|c|}{ ${t_i}$ } &   \multicolumn{1}{c|}{ ${\mbox{Ч}_i}$ } & 
\multicolumn{1}{c|}{ ${T_i}$ } \\ \hline

\multicolumn{1}{|c|}{\textbf{1}} &   \multicolumn{1}{c|}{\textbf{2}} & 
\multicolumn{1}{|c|}{\textbf{3}} &   \multicolumn{1}{c|}{\textbf{4}} & 
\multicolumn{1}{|c|}{\textbf{5}} &   \multicolumn{1}{c|}{\textbf{6}} & 
\multicolumn{1}{c|}{\textbf{7}} \\ \hline
\endfirsthead

\multicolumn{7}{|l|}{{Продолжение таблицы ~\ref{tab:eco2}}} \\ %\hline
\hline
\multicolumn{1}{|c|}{\textbf{1}} &   \multicolumn{1}{c|}{\textbf{2}} & 
\multicolumn{1}{c|}{\textbf{3}} & \multicolumn{1}{c|}{\textbf{4}} \\ \hline
\endhead

\endfoot

\hline
\endlastfoot

\multicolumn{7}{|c|}{\textbf{Техническое задание}} \\ \hline

1. & \begin{parbox}[c]{3.5cm}{Постановка задачи}\end{parbox} \\ \hline
2. & \begin{parbox}[c]{3.5cm}{Подбор литературы}\end{parbox} \\ \hline
3. & \begin{parbox}[b]{3.5cm}{Сбор исходных данных}\end{parbox} \\ \hline
4. & \begin{parbox}[t]{3.5cm}{Определение требований к системе}\end{parbox}
	\\ \hline
5. & \begin{minipage}{3.5cm}{Определение стадий, этапов и сроков разработки ПК}\end{minipage}
	\\ \hline

\multicolumn{7}{|c|}{\textbf{Эскизный проект}} \\ \hline

\hline

\end{longtable}
\end{center}

Таким образом, общая трудоёмкость разработки ПС составляет  88 человеко-дней, а их продолжительность – 112 календарных дней.

\subsubsection*{Построение ленточного графика разработки ПС}
В качестве инструмента планирования работ используем ленточный график. Ленточный график является удобным, простым и наглядным
инструментом для планирования работ. Он представляет собой таблицу, где перечислены  работы и длительность выполнения каждой
из них. Продолжением таблицы является линейный график, построенный в масштабе, наглядно показывающий продолжительность
каждой работы в виде отрезков прямых,  располагающихся в соответствии с последовательностью выполнения работ.

Ленточный график разработки ПС, построенный на основе данных табл. ~\ref{tab:eco2} приведён на рисунке ~\ref{pic:line_graphic} 
Он позволяет наглядно представить логическую последовательность и взаимосвязь отдельных работ, сроки начала и окончания работ,
соблюдение которых обеспечит своевременное выполнение проекта и разработку программных средств.

% Picture here

\subsection{Расчет сметы затрат на  разработку ПС}
Сметная стоимость проектирования и внедрения программы включает в себя  затраты, определяемые по формуле (\ref{eq:eco_smeta}):

\begin{equation}
C_{\mbox{пр}} = C_{\mbox{осн}} + C_{\mbox{доп}} + C_{\mbox{соц}} + C_{\mbox{м}} + C_{\mbox{маш.вр.}} + C_{\mbox{н}},
\label{eq:eco_smeta}
\end{equation}
где:	${C_{\mbox{пр}}}$ - стоимость разработки ПС; \\
	${C_{\mbox{м}}}$ - затраты на используемые материалы; \\
	${C_{\mbox{осн}}}$ - основная заработная плата исполнителей; \\
	${C_{\mbox{доп}}}$ - дополнительная заработная плата исполнителей, учитывающая потери времени на отпуска и болезни
		(принимается в среднем 10\% от основной заработной платы); \\
	${C_{\mbox{соц}}}$ - единый социальный налог (ЕСН), представляющий собой отчисления во внебюджетные фонды
		государственного социального страхования (пенсионный фонд, фонд обязательного медицинского страхования,
		фонд социального страхования). Рассчитывается в соответствии с установленной ставкой ЕСН как 26\% от
		основной и дополнительной заработной платы; \\
	${C_{\mbox{н}}}$ - накладные расходы включают затраты на управление, уборку, ремонт, электроэнергию, отопление и
		другие хозяйственные расходы (принимаются в размере 60\% от основной и дополнительной заработной платы);
	${C_{\mbox{маш.вр.}}}$ - стоимость машинного времени. \\

\subsubsection*{Основная заработная плата исполнителей}
На статью «Заработная плата» относят заработную плату научных, инженерно-технических и других работников,
непосредственно участвующих в разработке ПС. Расчёт ведётся по формуле (\ref{eq:eco_zarplata}):
\begin{equation}
\mbox{З}_{\mbox{исп}} = \mbox{З}_{\mbox{ср}} \cdot{T},
\label{eq:eco_zarplata}
\end{equation}
где:	${\mbox{З}_{\mbox{исп}}}$ - заработная плата исполнителей (руб.); \\
	${\mbox{З}_{\mbox{ср}}}$ -  средняя дневная тарифная ставка работника организации разработчика ПС (руб./чел.дни); \\
	${T}$ - трудоёмкость разработки ПС (чел.дни). \\

${\mbox{З}_{\mbox{ср}}}$ определяется по формуле (\ref{eq:eco_zsr}):
\begin{equation}
\mbox{З}_{\mbox{ср}} = \frac{C}{\mbox{Ф}_{\mbox{мес}}},
\label{eq:eco_zsr}
\end{equation}
где:	${C}$ - месячная зарплата работника (руб./мес.); \\
	${\mbox{Ф}_{\mbox{мес}}}$ - среднее количество рабочих дней в месяце (20дн.). \\

Расчёт затрат на основную заработную плату разработчиков ПС приведен в таблице \ref{tab:eco_zarplata}.

\begin{center}
\begin{longtable}{|c|c|c|c|c|c|}
\caption{Затраты на заработную плату} \label{tab:eco_zarplata} \\ \hline
\multicolumn{1}{|c|}{\textbf{№}} & \multicolumn{1}{c|}{\textbf{Исполнитель}} & 
\multicolumn{1}{p{2.5cm}|}{\textbf{Оклад, руб./мес.}} &   \multicolumn{1}{p{2.5cm}|}{\textbf{Оклад, руб./день}} & 
\multicolumn{1}{p{3.5cm}|}{\textbf{Трудоемкость, чел. дней}} & \multicolumn{1}{p{2cm}|}{\textbf{Сумма, руб.}} \\ \hline

\multicolumn{1}{|c|}{\textbf{1}} &   \multicolumn{1}{c|}{\textbf{2}} & 
\multicolumn{1}{c|}{\textbf{3}} &   \multicolumn{1}{c|}{\textbf{4}} & 
\multicolumn{1}{c|}{\textbf{5}} & \multicolumn{1}{c|}{\textbf{6}} \\ \hline
\endfirsthead

\multicolumn{6}{|l|}{{Продолжение таблицы ~\ref{tab:eco_zarplata}}} \\ %\hline
\hline
\multicolumn{1}{|c|}{\textbf{1}} &   \multicolumn{1}{c|}{\textbf{2}} & 
\multicolumn{1}{c|}{\textbf{3}} &   \multicolumn{1}{c|}{\textbf{4}} & 
\multicolumn{1}{c|}{\textbf{5}} & \multicolumn{1}{c|}{\textbf{6}} \\ \hline
\endhead

\endfoot

\hline
\endlastfoot

1. & Руководитель & 40000 & 2000 & 5 & 10000 \\ \hline
2. & Инженер & 30000 & 1500 & 75 & 112500 \\ \hline
 & Итого ${C_{\mbox{осн}}}$ & - & - & 80 & 122500 \\

\hline

\end{longtable}
\end{center}

\subsubsection*{Дополнительная заработная плата}
Дополнительная заработная плата на период разработки ПС рассчитывается относительно основной и составляет 10\% от её величины:

\begin{equation}
C_{\mbox{доп}} = C_{\mbox{осн}} \cdot 0.1 = 122500 \cdot 0.1 = 12250,
%\label{eq:eco_smeta}
\end{equation}

\subsubsection*{Расчёт отчислений на социальное страхование}
Отчисления на социальное страхование рассчитываются относительно выплаченной заработной платы (суммы основной и
дополнительной заработной платы). Составляют 26\%:

\begin{equation}
C_{\mbox{соц}} = (C_{\mbox{доп}} + C_{\mbox{осн}}) \cdot 0.26 = (122500 + 12250) \cdot 0.26 = 35035 (\mbox{руб}),
%\label{eq:eco_smeta}
\end{equation}

\subsubsection*{Расчёт расходов на материалы}
На эту статью относят все затраты на магнитные носители данных, бумагу, для печатных устройств, канцтовары и др.
Затраты по ним определяются по экспертным оценкам. Расчёт расходов на материалы приведён в табл. \ref{tab:eco_zatnm}

\begin{center}
\begin{longtable}{|c|c|c|c|}
\caption{Затраты на заработную плату} \label{tab:eco_zatnm} \\ \hline
\multicolumn{1}{|c|}{\textbf{№}} & \multicolumn{1}{c|}{\textbf{Материалы}} & 
\multicolumn{1}{c|}{\textbf{Кол-во, шт}} &   \multicolumn{1}{c|}{\textbf{Стоимость, руб.}} \\ \hline

\multicolumn{1}{|c|}{\textbf{1}} &   \multicolumn{1}{c|}{\textbf{2}} & 
\multicolumn{1}{c|}{\textbf{3}} &   \multicolumn{1}{c|}{\textbf{4}} \\ \hline
\endfirsthead

\multicolumn{4}{|l|}{{Продолжение таблицы ~\ref{tab:eco_zatnm}}} \\ %\hline
\hline
\multicolumn{1}{|c|}{\textbf{1}} &   \multicolumn{1}{c|}{\textbf{2}} & 
\multicolumn{1}{c|}{\textbf{3}} &   \multicolumn{1}{c|}{\textbf{4}} \\ \hline
\endhead

\endfoot

\hline
\endlastfoot

1. & Блокноты & 2 & 70  \\ \hline
2. & Ручки & 10 & 30  \\ \hline
3. & Другие канцтовары & - & 1900 \\ \hline
 & Итого ${C_{\mbox{м}}}$ & & 2000 \\

\hline

\end{longtable}
\end{center}

\subsubsection*{Накладные расходы}
На статью «Накладные расходы» относят расходы, связанные с управлением и организацией работ, содержанием помещений
(освещение, отопление, уборка и т.д.). Накладные расходы рассчитываются относительно основной заработной платы. Величина
накладных расходов принимается равной 60\% от основной зарплаты исполнителей. Формула расчёта (\ref{eq:eco_naklr})

\begin{equation}
C_{\mbox{н}} = C_{\mbox{осн}} \cdot K,
\label{eq:eco_naklr}
\end{equation}
где:	${C_{\mbox{н}}}$ - накладные расходы (руб.); \\
	${C_{\mbox{осн}}}$ - основная заработная плата исполнителей (руб.);
	${K}$ - коэффициент учёта накладных расходов (К=0.6).

\begin{center}
${C_{\mbox{н}} = 122500 \cdot 0.6 = 73500 (\mbox{руб.})}$
\end{center}

\subsubsection*{Расчёт стоимости машинного времени}
Включает затраты на машинное время, необходимое для разработки ПС, расходы на подготовку и приобретение материалов
научно-технической информации, расходы на использование средств связи. Расчет затрат на машинное время осуществляется
по формуле (\ref{eq:eco_pricePC}):

\begin{equation}
C_{\mbox{маш.вр.}} = K_{\mbox{маш.вр.}} \cdot 3_{\mbox{маш.вр.}},
\label{eq:eco_pricePC}
\end{equation}
где:	${K_{\mbox{маш.вр.}}}$ - средняя стоимость одного часа машинного времени (берется 50руб./час); \\
	${3_{\mbox{маш.вр.}}}$ - машинное время, используемое на проведение исследования.	

Необходимое количество машинного времени для реализации проекта по разработке программы рассчитывается по формуле (\ref{eq:eco_zmashvr}):

\begin{equation}
3_{\mbox{маш.вр.}} = t_i \cdot T_{\mbox{см}} \cdot T_{\mbox{ср.маш.}},
\label{eq:eco_zmashvr}
\end{equation}
где:	${t_i}$ - трудоёмкость работ, человек-дней;
	${T_{\mbox{см}}}$ - продолжительность рабочей смены (при пятидневной рабочей неделе - 8 часов);
	${T_{\mbox{ср.маш.}}}$ - средний коэффициент использования оборудования (берется 0.7).

Тогда:
\begin{center}
${3_{\mbox{маш.вр.}} = 75 \cdot 8 \cdot 0.7 = 420}$ (маш.час)
\end{center}

Стоимость машинного времени составит:
\begin{center}
${C_{\mbox{маш.вр.}} = 50 \cdot 420 = 21000}$ (руб.)
\end{center}

Результаты расчета затрат на проектирование программных средств представлены в таблице \ref{tab:eco_smeta}.

\begin{center}
\begin{longtable}{|c|c|c|c|c|}
\caption{Затраты на заработную плату} \label{tab:eco_smeta} \\ \hline
\multicolumn{1}{|c|}{\textbf{№}} & \multicolumn{1}{c|}{\textbf{Наименование статей}} & 
\multicolumn{1}{c|}{\textbf{Обозначение}} &   \multicolumn{1}{c|}{\textbf{Сумма, руб}} & 
\multicolumn{1}{c|}{\textbf{В \% к итогу}} \\ \hline

\multicolumn{1}{|c|}{\textbf{1}} &   \multicolumn{1}{c|}{\textbf{2}} & 
\multicolumn{1}{c|}{\textbf{3}} &   \multicolumn{1}{c|}{\textbf{4}} & 
\multicolumn{1}{c|}{\textbf{5}} \\ \hline
\endfirsthead

\multicolumn{5}{|l|}{{Продолжение таблицы ~\ref{tab:eco_smeta}}} \\ %\hline
\hline
\multicolumn{1}{|c|}{\textbf{1}} &   \multicolumn{1}{c|}{\textbf{2}} & 
\multicolumn{1}{c|}{\textbf{3}} &   \multicolumn{1}{c|}{\textbf{4}} & 
\multicolumn{1}{c|}{\textbf{5}} \\ \hline
\endhead

\endfoot

\hline
\endlastfoot

1. & Основная заработная плата & ${C_{\mbox{осн.}}}$ & 122500 & 45.4 \\ \hline
2. & Доп. заработная плата & ${C_{\mbox{доп.}}}$ & 12250 & 4.6 \\ \hline
3. & Отчисления на соц. нужды & ${C_{\mbox{соц.}}}$ & 35035 & 12.25 \\ \hline
4. & Материалы & ${C_{\mbox{мат.}}}$ & 2000 & 0.75 \\ \hline
5. & Стоимость маш. времени & ${C_{\mbox{маш.вр.}}}$ & 21000 & 9 \\ \hline
6. & Накладные расходы & ${C_{\mbox{н}}}$ & 73500 & 28 \\ \hline
   & Итого: & ${C_{\mbox{пр}}}$ & 266285 & 100 \\ 

\hline

\end{longtable}
\end{center}

Таким образом, себестоимость разработки ПС составляет 266285 руб.

Данный ПК может быть реализован на рынке. При расчётном количестве реализованных программ – n, оптовая цена программы
(${\mbox{Ц}_{\mbox{опт}}}$) может быть рассчитана по формуле:

\begin{equation}
\mbox{Ц}_{\mbox{опт}} = \frac{\mbox{С}_{\mbox{пр}}}{n} + \mbox{П}_i,
%\label{eq:eco_zarplata}
\end{equation}
где:	${\mbox{С}_{\mbox{пр}}}$ - себестоимость разработки программы; \\
	${\mbox{П}}$ - прибыль, определяется по формуле:

\begin{equation}
\mbox{П}_i = Y_p \cdot \frac{C_{npi}}{n} \cdot 100,
%\label{eq:eco_zarplata}
\end{equation}
где:	${Y_p}$ -  средний уровень рентабельности (принимается Yр= 20\%);\\

Таким образом, при среднем расчётном количестве реализованных ПК n = 10 оптовая цена ПС составит: \\
\begin{center}
${\mbox{Ц}_{\mbox{опт}} = \frac{266285}{10} + 0.2 \cdot \frac{266285}{10} = 31955}$ руб. \\
\end{center}

Отпускная оптовая цена реализации программы потребителям должна включать налог на добавленную
стоимость (НДС) и рассчитывается по формуле:

\begin{equation}
\mbox{Ц}_{\mbox{отп}} = \mbox{Ц}_{\mbox{опт}} + \mbox{НДС},
%\label{eq:eco_zarplata}
\end{equation}
где НДС – налог на добавленную стоимость,  рассчитывается в соответствии с действующей ставкой этого
налога - 18\% от оптовой цены программы:

\begin{center}
${\mbox{НДС} = 31955 \cdot 0.18 = 5752\mbox{(руб.)}}$ \\
${\mbox{Ц}_{\mbox{отп}} = 31955 + 5752 = 32707 \mbox{(руб.)}}$
\end{center}

Таким образом, отпускная цена программы составит  32707 руб.,  в том числе НДС – 5752 (руб.).

\subsection{Расчет основных технико-экономических показателей использования программного продукта}
В настоящей дипломной работе  разработаны алгоритмы и проведена программная реализация эксперимента по
автоматизации защиты локальной вычислительной сети. Использование этих алгоритмов и программных средств 
позволит повысить эффективность защиты локальной вычислительной сети, соответственно повысится эффективность
работы администратора, следящего за бесперебойной работой сети, что способствует более качественному уровню
обслуживания пользователей.

Основные технико-экономические показатели проведения исследования приведены в таблице \ref{tab:eco_osn_tepp} 

\begin{center}
\begin{longtable}{|c|c|c|c|}
\caption{Затраты на заработную плату} \label{tab:eco_osn_tepp} \\ \hline
\multicolumn{1}{|c|}{\textbf{№}} & \multicolumn{1}{c|}{\textbf{Наименование показателя}} & 
\multicolumn{1}{p{3cm}|}{\textbf{Единица измерения}} &   \multicolumn{1}{c|}{\textbf{Проектный вариант}} \\ \hline 

\multicolumn{1}{|c|}{\textbf{1}} &   \multicolumn{1}{c|}{\textbf{2}} & 
\multicolumn{1}{c|}{\textbf{3}} & \multicolumn{1}{c|}{\textbf{4}} \\ \hline
\endfirsthead

\multicolumn{4}{|l|}{{Продолжение таблицы ~\ref{tab:eco_osn_tepp}}} \\ %\hline
\hline
\multicolumn{1}{|c|}{\textbf{1}} &   \multicolumn{1}{c|}{\textbf{2}} & 
\multicolumn{1}{c|}{\textbf{3}} & \multicolumn{1}{c|}{\textbf{4}} \\ \hline
\endhead

\endfoot

\hline
\endlastfoot

1. & Способ обработки информации & - & С применением ЭВМ и ПС  \\ \hline
2. & Хар-ки исследования & - &   \\ \hline
3. & Языки программирования & - & VHDL, C, Bash  \\ \hline
4. & Технические ср-ва & & \\ \hline
   & ПК & G6950, 2.8 Ghz  \\ \hline
5. & Кол-во исследователей & чел & 1  \\ \hline
6. & Прод-ть исследования & кал. дней & XXX  \\ \hline
7. & Трудоемкость & чел-дней & 88  \\ \hline
8. & Затраты & руб & 266285  \\ \hline
   & в том числе: &  & \\ \hline
   & расх. мат. & руб & 2000 \\ \hline
   & осн. зп.  & руб & 122500 \\ \hline
   & доп. зп.  & руб & 12250 \\ \hline
   & соц. нужды  & руб & 35035 \\ \hline
   & накладные расходы & руб & 73500 \\ \hline
   & стоимость маш. времени & руб &  21000\\ \hline

\end{longtable}
\end{center}

\newpage
\section{Выводы}
В организационно-экономическом разделе определены стадии разработки ПС, состав работ, рассчитано время, требующееся на
проведение исследования и тестирование, построен ленточный график разработки ПС, определены затраты на разработку ПС,
приведены основные технико-экономические показатели проведения исследования.
Трудоемкость разработки, согласно расчетам, составит 88 человеко-дней, продлится 112 календарных дня, а затраты на
нее составят 221798 рублей.

\newpage

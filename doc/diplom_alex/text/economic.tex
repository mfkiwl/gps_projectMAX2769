\section{ОРГАНИЗАЦИОННО-ЭКОНОМИЧЕСКИЙ РАЗДЕЛ}
\subsection{Планирование разработки программных средств с построением графика}
Целью дипломного проекта является разработка программного комплекса (ПК) оценки трудоемкости объектно-ориентированных программ.
В данном разделе определяется трудоёмкость и затраты на создание ПК, а так же производится расчёт основных технико-экономических
показателей проекта.

\subsubsection*{Определение трудоемкости и продолжительности работ по созданию ПК}
Процесс разработки включает: обзор и анализ программных средств схожей тематики, анализ и выбор программных продуктов для
создания программы; отладка; испытание. В свою очередь каждый из этих этапов можно подразделить на отдельные под этапы.
Согласно ГОСТ 23501.1-79 регламентируются следующие стадии проведения исследования:

\begin{itemize}
	\item техническое задание – ТЗ (ГОСТ 23501.2-79);
	\item эскизный проект – ЭП (ГОСТ 23501.5-80);
	\item технический проект – ТП (ГОСТ 23501.6-80);
	\item рабочий проект – РП (ГОСТ 23501.11-81);
	\item внедрение – ВП (ГОСТ 23501.15-81).
\end{itemize}

Планирование стадий и содержания работ осуществляется в соответствии с \cite{bibl51}. На всех стадиях проведения исследования
выполняются следующие виды работ, перечень которых показан в таблице ~\ref{tab:eco1}.

% http://users.sdsc.edu/~ssmallen/latex/longtable.html
\begin{center}
\begin{longtable}{|l|l|}
\caption{Состав работ и стадии разработки ПК} \label{tab:eco1} \\ \hline
\multicolumn{1}{|c|}{\textbf{Стадии разработки}}    &   \multicolumn{1}{c|}{\textbf{Перечень работ}} \\ \hline
\multicolumn{1}{|c|}{\textbf{1}}    &   \multicolumn{1}{|c|}{\textbf{2}} \\ \hline
\endfirsthead

%\multicolumn{2}{c} %
%{{\bfseries \tablename \thetable{} -- continued from previous page}} \\
\multicolumn{2}{|l|}{{Продолжение таблицы ~\ref{tab:eco1}}} \\ %\hline
\hline \multicolumn{1}{|c|}{\textbf{1}} &
%\multicolumn{1}{c|}{\textbf{Triple chosen}} &
\multicolumn{1}{c|}{\textbf{2}} \\ \hline 
\endhead

%\multicolumn{2}{|r|}{{}} \\ %\hline
%\hline \multicolumn{2}{|r|}{{Continued on next page}} \\ %\hline
\endfoot

\hline
\endlastfoot
		Техническое задание & \begin{parbox}{5in} {
						\begin{itemize}
							\item постановка задачи;
							\item подбор литературы;
							\item сбор исходных данных;
							\item определение требований к системе;
							\item определение стадий, этапов и сроков разработки ПК;
						\end{itemize} }  
					\end{parbox} \\
	\hline
		Эскизный проект & \begin{parbox}{5in} {
					\begin{itemize} 
						\item анализ программных средств схожей тематики;
						\item разработка общей структуры ПК;
						\item разработка структуры программы по подсистемам;
						\item документирование;
					\end{itemize} }
				\end{parbox} \\
	\hline
		Технический проект & \begin{parbox}{5in} {
					\begin{itemize} 
						\item определение требований к ПК;
						\item выбор инструментальных средств;
						\item определение свойств и требований к аппаратному обеспечению;
					\end{itemize} } 
				\end{parbox} \\
	\hline
		Рабочий проект & \begin{parbox}{5in} {
					\begin{itemize} 
						\item программирование;
						\item тестирование и отладка ПК;
						\item разработка программной документации;
						\item согласование и утверждение программы и методики испытаний;
					\end{itemize} }
				\end{parbox} \\
	\hline
		Внедрение & \begin{parbox}{5in} {
					\begin{itemize}
						\item ;
						\item ;
						\item ;
						\item ;
					\end{itemize} }
				\end{parbox} \\
\end{longtable}
\end{center}

Трудоемкость выполнения работ по созданию ПК  на каждой из стадий определяется в соответствии с \cite{bibl52, bibl53}.

Трудоемкость разработки ПК определяется по сумме трудоемкости этапов и видов работ, оцениваемых экспертным путем в
человеко-днях, и носит вероятностный характер, так как зависит от множества трудно учитываемых факторов.

Трудоемкость каждого вида работ определяется в соответствии с методическими указаниями \cite{bibl53} по формуле:

\begin{equation}
t_i = \frac{3\cdot{t_{min}} + 2\cdot{t_{max}}}{5},
\label{eq:eco1}
\end{equation}
где:	$t_{min}$ - минимально возможная трудоемкость выполнения отдельного вида работ в человеко-днях; \\
	$t_{max}$ - максимально возможная трудоемкость выполнения отдельного вида работ в человеко-днях.

Продолжительность каждого вида работ в календарных днях ${(T_i)}$ определяется по формуле \cite{bibl53}:

\begin{equation}
T_i = \frac{t_i}{\mbox{Ч}_i}\cdot{K_{\mbox{вых}}},
\label{eq:eco2}
\end{equation}
где:	${t_i}$ - трудоемкость работы, человек-дней; \\
	${\mbox{Ч}_i}$ - численность исполнителей, человек; \\
	${K_{\mbox{вых}}}$ - коэффициент, учитывающий выходные и праздничные дни: \\

\begin{equation}
T_i = \frac{K_{\mbox{кап}}}{K_{\mbox{раб}}},
\label{eq:eco3}
\end{equation}
где:	${K_{\mbox{кап}}}$ - число календарных дней;
	${K_{\mbox{раб}}}$ - рабочие дни.

Для расчёта принимается среднее значение ${K_{\mbox{вых}} = 1.45}$. 

Полный список видов и этапов работ по созданию ПС, экспертные оценки и расчетные величины их трудоемкости,
а также продолжительность каждого вида работ, рассчитанные по формулам (\ref{eq:eco1}) и (\ref{eq:eco2}) представлены
в таблице ~\ref{tab:eco2}.

% table

%\begin{center}
%\begin{longtable}{|l|l|}
%\caption{Расчёт трудоёмкости и продолжительности работ по созданию ПС} \label{tab:eco2} \\ \hline

%\multicolumn{1}{|p{1cm}|}{\textbf{ 1 }} &
%\multicolumn{1}{|p{5cm}|}{\textbf{2}} &
%\multicolumn{1}{|p{5cm}|}{\textbf{3}} \\ \hline
%& \multicolumn{1}{|p{1cm}|}{\textbf{Количество исполнителей, чел.}} \\ \hline
%\multicolumn{1}{|c|}{\textbf{Продолжительность, календарные дни}} \\ \hline

%\multicolumn{1}{|c|}{\textbf{1}}    &   \multicolumn{1}{c|}{\textbf{2}} \\ \hline

%\endfirsthead

%\multicolumn{2}{|l|}{{Продолжение таблицы ~\ref{tab:eco1}}} \\ %\hline
%\hline \multicolumn{1}{|c|}{\textbf{1}} &
%\multicolumn{1}{c|}{\textbf{2}} \\ \hline 
%\endhead
%\endfoot

%\hline
%\endlastfoot

% ================================================================

%\end{longtable}
%\end{center}

% end of the table

Таким образом, общая трудоёмкость разработки ПС составляет  88 человеко-дней, а их продолжительность – 112 календарных дней.

\subsubsection*{Построение ленточного графика разработки ПС}
В качестве инструмента планирования работ используем ленточный график. Ленточный график является удобным, простым и наглядным
инструментом для планирования работ. Он представляет собой таблицу, где перечислены  работы и длительность выполнения каждой
из них. Продолжением таблицы является линейный график, построенный в масштабе, наглядно показывающий продолжительность
каждой работы в виде отрезков прямых,  располагающихся в соответствии с последовательностью выполнения работ.

Ленточный график разработки ПС, построенный на основе данных табл. ~\ref{tab:eco2} приведён на рисунке ~\ref{pic:line_graphic} 
Он позволяет наглядно представить логическую последовательность и взаимосвязь отдельных работ, сроки начала и окончания работ,
соблюдение которых обеспечит своевременное выполнение проекта и разработку программных средств.

% Picture here

\subsection{Расчет сметы затрат на  разработку ПС}
Сметная стоимость проектирования и внедрения программы включает в себя  затраты, определяемые по формуле (\ref{eq:eco_smeta}):

\begin{equation}
C_{\mbox{пр}} = C_{\mbox{осн}} + C_{\mbox{доп}} + C_{\mbox{соц}} + C_{\mbox{м}} + C_{\mbox{маш.вр.}} + C_{\mbox{н}},
\label{eq:eco_smeta}
\end{equation}
где:	${C_{\mbox{пр}}}$ - стоимость разработки ПС; \\
	${C_{\mbox{м}}}$ - затраты на используемые материалы; \\
	${C_{\mbox{осн}}}$ - основная заработная плата исполнителей; \\
	${C_{\mbox{доп}}}$ - дополнительная заработная плата исполнителей, учитывающая потери времени на отпуска и болезни
		(принимается в среднем 10\% от основной заработной платы); \\
	${C_{\mbox{соц}}}$ - единый социальный налог (ЕСН), представляющий собой отчисления во внебюджетные фонды
		государственного социального страхования (пенсионный фонд, фонд обязательного медицинского страхования,
		фонд социального страхования). Рассчитывается в соответствии с установленной ставкой ЕСН как 26\% от
		основной и дополнительной заработной платы; \\
	${C_{\mbox{н}}}$ - накладные расходы включают затраты на управление, уборку, ремонт, электроэнергию, отопление и
		другие хозяйственные расходы (принимаются в размере 60\% от основной и дополнительной заработной платы);
	${C_{\mbox{маш.вр.}}}$ - стоимость машинного времени. \\

\subsubsection*{Основная заработная плата исполнителей}
На статью «Заработная плата» относят заработную плату научных, инженерно-технических и других работников,
непосредственно участвующих в разработке ПС. Расчёт ведётся по формуле (\ref{eq:eco_zarplata}):
\begin{equation}
\mbox{З}_{\mbox{исп}} = \mbox{З}_{\mbox{ср}} \cdot{T},
\label{eq:eco_zarplata}
\end{equation}
где:	${\mbox{З}_{\mbox{исп}}}$ - заработная плата исполнителей (руб.); \\
	${\mbox{З}_{\mbox{ср}}}$ -  средняя дневная тарифная ставка работника организации разработчика ПС (руб./чел.дни); \\
	${T}$ - трудоёмкость разработки ПС (чел.дни). \\

${\mbox{З}_{\mbox{ср}}}$ определяется по формуле (\ref{eq:eco_zsr}):
\begin{equation}
\mbox{З}_{\mbox{ср}} = \frac{C}{\mbox{Ф}_{\mbox{мес}}},
\label{eq:eco_zsr}
\end{equation}
где:	${C}$ - месячная зарплата работника (руб./мес.); \\
	${\mbox{Ф}_{\mbox{мес}}}$ - среднее количество рабочих дней в месяце (20дн.). \\

Расчёт затрат на основную заработную плату разработчиков ПС приведен в таблице \ref{tab:eco_zarplata}.



\newpage

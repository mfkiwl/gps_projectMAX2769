%\documentclass[a4paper,12pt]{eskdtext}		%размер бумаги устанавливаем А4, шрифт 12пунктов
\documentclass[a4paper,14pt]{scrartcl}		%размер бумаги устанавливаем А4, шрифт 12пунктов
%\documentclass[14pt,a4paper,twoside]{report}
\usepackage[T2A]{fontenc}
\usepackage[utf8]{inputenc}			%включаем свою кодировку: koi8-r или utf8 в UNIX, cp1251 в Windows
\usepackage[english,russian]{babel}		%используем русский и английский языки с переносами
\usepackage{amssymb,amsfonts,amsmath,mathtext,cite,enumerate,float} %подключаем нужные пакеты расширений
\usepackage[dvips]{graphicx}			%хотим вставлять в диплом рисунки?
\usepackage{soul}
\usepackage[14pt]{extsizes}
\usepackage{longtable} 
\usepackage{graphicx}

\graphicspath{{images/}}			%путь к рисункам

\linespread{1.5}

\makeatletter
\renewcommand{\@biblabel}[1]{#1.} 		% Заменяем библиографию с квадратных скобок на точку:
\makeatother

\RequirePackage{lscape}

%\renewcommand\large{\@setfontsize\large{15.5}{17}}
%\renewcommand\Large{\@setfontsize\Large{16.5}{19}}
%\renewcommand\small{\@setfontsize\small{9}{9.5}}

\usepackage{geometry} 				% Меняем поля страницы
\geometry{left=2cm}				% левое поле
\geometry{right=1.4cm}				% правое поле
\geometry{top=1cm}				% верхнее поле
\geometry{bottom=2cm}				% нижнее поле

\renewcommand{\theenumi}{\arabic{enumi}}	% Меняем везде перечисления на цифра.цифра
\renewcommand{\labelenumi}{\arabic{enumi}}	% Меняем везде перечисления на цифра.цифра
\renewcommand{\theenumii}{\arabic{enumii}}	% Меняем везде перечисления на цифра.цифра
\renewcommand{\labelenumii}{\arabic{enumi}.\arabic{enumii}.}% Меняем везде перечисления на цифра.цифра
\renewcommand{\theenumiii}{\arabic{enumiii}}	% Меняем везде перечисления на цифра.цифра
\renewcommand{\labelenumiii}{\arabic{enumi}.\arabic{enumii}.\arabic{enumiii}.}% Меняем везде перечисления на цифра.цифра

% redefine title and bibl
\addto\captionsrussian{\def\contentsname{Содержание}}
%\addto\captionsrussian{\def\bibname{Список rrrrrлитературы}}
\addto\captionsrussian{\def\refname{Список используемых источников}}

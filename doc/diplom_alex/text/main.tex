%\documentclass[a4paper,12pt]{eskdtext}		%размер бумаги устанавливаем А4, шрифт 12пунктов
\documentclass[a4paper,14pt]{scrartcl}		%размер бумаги устанавливаем А4, шрифт 12пунктов
\usepackage[T2A]{fontenc}
\usepackage[utf8]{inputenc}			%включаем свою кодировку: koi8-r или utf8 в UNIX, cp1251 в Windows
\usepackage[english,russian]{babel}		%используем русский и английский языки с переносами
\usepackage{amssymb,amsfonts,amsmath,mathtext,cite,enumerate,float} %подключаем нужные пакеты расширений
\usepackage[dvips]{graphicx}			%хотим вставлять в диплом рисунки?
\graphicspath{{images/}}			%путь к рисункам

\makeatletter
\renewcommand{\@biblabel}[1]{#1.} 		% Заменяем библиографию с квадратных скобок на точку:
\makeatother

\usepackage{geometry} 				% Меняем поля страницы
\geometry{left=2cm}				% левое поле
\geometry{right=1.5cm}				% правое поле
\geometry{top=1cm}				% верхнее поле
\geometry{bottom=2cm}				% нижнее поле

\renewcommand{\theenumi}{\arabic{enumi}}	% Меняем везде перечисления на цифра.цифра
\renewcommand{\labelenumi}{\arabic{enumi}}	% Меняем везде перечисления на цифра.цифра
\renewcommand{\theenumii}{\arabic{enumii}}	% Меняем везде перечисления на цифра.цифра
\renewcommand{\labelenumii}{\arabic{enumi}.\arabic{enumii}.}% Меняем везде перечисления на цифра.цифра
\renewcommand{\theenumiii}{\arabic{enumiii}}	% Меняем везде перечисления на цифра.цифра
\renewcommand{\labelenumiii}{\arabic{enumi}.\arabic{enumii}.\arabic{enumiii}.}% Меняем везде перечисления на цифра.цифра

\begin{document}
\begin{titlepage}
\newpage

\begin{center}
Федеральное агентство по образованию \\
\vspace{1cm}
\small{Государственное образовательное учреждение высшего профессионального образования} \\
\normalsize{“Московский государственный университет приборостроения и информатики”} \\*
%\hrulefill
\end{center}

\begin{center}
Факультет ИТ  специальность (направление) 230105 \\
Кафедра ИТ-6  квалификация (степень) \\
\end{center}

\begin{flushright}
\textbf{Утверждаю} \\
Зав. кафедрой \\ 
«»       2010г.
\end{flushright}

\vfill

\begin{center}
\textbf{\Large ПОЯСНИТЕЛЬНАЯ ЗАПИСКА \\
к дипломному проекту на тему: }
\end{center}

\vfill\vfill

\begin{center}
\textsc{\textbf{ничо не секу\linebreak ваще}}
\end{center}

\vfill\vfill

\begin{flushleft}
Дипломник \hrulefill Никифоров А.А \\
Группа \hrulefill   шифр \hrulefill \\
Обозначение проекта (работы) \hrulefill \\
\vfill
\textbf{Руководитель проекта (работы) \hrulefill}
\vfill
\begin{center}
Консультанты по разделам:
\end{center}
Наименование разделов: \\
\hrulefill \\
\hrulefill \\
\hrulefill \\
Нормоконтроль \hrulefill \\
\end{flushleft}

\vspace{\fill}

\begin{center}
Москва 2010г.
\end{center}

\end{titlepage}
					% это титульный лист
\tableofcontents 				% это оглавление, которое генерируется автоматически
\section{ТЕХНОЛОГИЧЕСКИЙ РАЗДЕЛ}

\subsection{Описание аппаратной платформы для приема сигналов GPS}

\begin{figure}[H]
\center{\includegraphics[width=1\linewidth]{./pics/board_scheme.eps}}
\caption{Системы захвата данных}
\label{pic:board_scheme}
\end{figure}

\subsection{Описание разработанных встраиваемых решений (VHDL)}

% ===================== OLD =================================
\subsubsection{RS-232}
\label{razdel3_rs232}
Для реализации интерфейса RS232 

\subsubsection{Модули для SRAM-микросхемы M5M5V208FP-85}
\label{razdel3_sram}

\subsubsection{Модули для GPS микросхемы MAX2769}
\label{razdel3_gps}

\subsection{Контроллер для SRAM-микросхемы M5M5V208FP-85}
\label{sec:sram_controller}
%\begin{figure}[h]
%\center{\includegraphics[width=1\linewidth]{./pics/gps_serial_clock_oscylloscope.eps}}
%\caption{Clock-сигнал для serial-интерфейса GPS}
%\end{figure}

\subsection{Описание прикладного программного обеспечения}

\newpage
					% технологический раздел 
\end{document}

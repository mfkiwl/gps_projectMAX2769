%\addcontentsline{toc}{chapter}{Список используемых источников}

\begin{thebibliography}{99}
% 1.1
\bibitem{yacenkov} Яценков В.С. Основы спутниковой навигации. Системы GPS NAVSTAR и ГЛОНАСС.-М:Горячая линия-Телеком, 2005.-272 с.:ил.
\bibitem{tsui} Bao-Yen Tsui, J., Fundamentals of Global Positioning System Receivers: A Software Approach, 2nd ed. (New York: John Wiley \& Sons, Inc., 2004).
\bibitem{gpsyellow} Grewal, M. S., Weill, L. R., and Andrews, A. P., Global Positioning Systems, Inertial Navigation, and Integration, John Wiley \& Sons, New York, NY, 2001.
\bibitem{gps} Borre, Kai, Akos, Dennis, Bertelsen, Nicolaj, Rinder, Peter \& Holdt Jensen, Soren (2007) A Software-Defined GPS and Galileo Receiver. Single-Frequency Approach. Birkhauser Boston, 200 pages

%1.2
\bibitem{ni_acq} http://www.ni.com/dataacquisition/whatis.htm.
\bibitem{ni_article} http://zone.ni.com/devzone/cda/tut/p/id/7189
\bibitem{seamax_overview} http://www.seasolve.com/mobile-wimax-signal-analyzer.html
\bibitem{seamax_pdf} http://www.seasolve.com/pdf/wimax-brochure.pdf
\bibitem{soft_gps} http://kom.aau.dk/project/softgps/
\bibitem{soft_gps1} http://www.nsl.eu.com/primo\_order.php

% 2 special
\bibitem{boyd} Duncan Boyd,  Calculate, The Uncertainty Of NF Measurements. Microwave \& RF. Magazine. pp 93-102


% 3 tech 

% section 4 - BJD
% 4.1
\bibitem{npb10503} НПБ 105-03. Определение категорий помещений, зданий и наружных установок по
взрывопожарной и пожарной опасности. - М.: ГУ ГПС МДВ РФ,2003.

% 4.2
\bibitem{bjd421} ГОСТ 12.3.046—91 ССБТ. Установки пожаротушения автоматические. Общие технические требования 
\bibitem{bjd422} ПУЭ-03 "Правила устройства электроустановок"

% 4.3
\bibitem{bjd43} ГОСТ 21930-76. Припои оловянно-свинцовые в чушках. Технические условия.

% 5 eco
\bibitem{bibl51} "Методические указания по выполнению курсовой работы для студентов специальности 22.01.05
	на тему "Организация, планирование и управление предприятием машиностроительной промышленности" М.,МГАПИ, 2003.
\bibitem{bib51} Чаплыгин В.А. Организационно-экономический раздел дипломных работ научно-исследовательского характера. М., МГАПИ. 2000.
\bibitem{bibl52} Методические указания по сбору материалов на преддипломной практике и выполнению организационно-экономического
	раздела дипломных проектов. М., МГАПИ, 2004.
\bibitem{bibl53} Капелюш Г.С., Шестоперов С.Б. Технико-экономическое обоснование дипломных проектов по созданию программных средств
	вычислительной техники и информатики. Учебное пособие для студентов специальности 22.01. – М., МГАПИ, 2001.
\bibitem{bibl54} «Сетевые графики и планирование», учебное пособие, Н. И. Новицкий, М., Высшая школа, 2004.
\bibitem{bibl55} Прайс-лист ООО “НеоТорг” 28.02.2010г.

% misc
%\bibitem{sram} M5M5V208FP,VP,RV,KV,KR data sheet
%\bibitem{gps_max} MAX2769 data sheet
%\bibitem {fast_mem_test} http://www.netrino.com/Embedded-Systems/How-To/Memory-Test-Suite-C

\end{thebibliography}

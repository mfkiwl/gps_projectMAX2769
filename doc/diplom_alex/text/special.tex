\section{СПЕЦИАЛЬНЫЙ РАЗДЕЛ}

%--------------------------------------------------------------------------------
\subsection{Математическая модель преобразований сигнала в приемнике}
GPS приемник состоит из RF-части, IF-части и части обработки сигнала (аппаратно или программно ориентированной). RF (radio frequency)
состоит из всех компонентов до первого смесителя (рисунок \ref{pic:hw_receiver}). На приведенном рисунке в нее входят: антенна,
усилитель 1 (входит в состав активной антенны), bias-tee компонент (необходим для подачи питания с цифрового источника в аналоговый
кабель антенны для усилителя 1) и широкополосный фильтр. IF (intermidiate frequency) состоит из всех компонентов после первого
смесителя до АЦП или второго смесителя, в некоторых конструкциях приемников \cite{gps}.

\begin{figure}[h]
\begin{center}
	\scalebox{0.9}{\includegraphics[width=1\linewidth]{./pics/gps_receiver.eps}}
\end{center}
\caption{Аппаратная часть GNSS L1 приемника}
\label{pic:hw_receiver}
\end{figure}

Первым компонентом после антенны может быть усилитель (как на рисунке \ref{pic:hw_receiver}) или широкополосный фильтр. Оба
варианта имеют достоинства и недостатки.

Формула для вычисления коэффициента шума (noise figure), называется формулой Фрииса \cite{boyd} и может быть записана как:

\begin{equation}
F_{\mbox{системы}} = 
			F_1 + \frac{F_2 - 1}{G_1} + \frac{F_3 - 1}{G_1 G_2} + ... + \frac{F_N - 1}{G_1 G_2 ... G_{N1}},
		   =
			\frac{SNR_{in}}{SNR_{out}}
\label{eq:friis}
\end{equation}
где ${F_i}$ и ${G_i (i=1,2,...N)}$ - коэффициенты шума и усиления каждого компонента в RF цепи.

Если первым элементом является усилитель, то коэффициент шума приемника будет приблизительно равным коэффициенту шума
усилителя 1 (рисунок \ref{pic:hw_receiver}), который может быть менее 2 дБ. Влияние следующего элемента RF цепи, например 
широкополосного усилителя, на коэффициент шума будет снижен на коэффициент усиления усилителя 1 (рисунок \ref{pic:hw_receiver}).
Потенциальной проблемой данного подхода является то, что при сильном уровне сигнала полоса пропускания может достичь насыщения
и начать генерировать побочные частоты.

Основной задачей смесителя на рисунке \ref{pic:hw_receiver} является перевод сигнала с высокой (RF) частоты на более низкую (IF)
частоту. GPS L1 сигнал поступает на частоте 1575.42 МГц. Основной целью данной операции явлеяется перевод сигнала
в удобный для работы диапазон частот, в частности для подачи на АЦП. В дизайне, отображенном на рисунке \ref{pic:hw_receiver},
отражен один каскад понижения частоты, однако их может быть несколько \cite{gps}. 

% software gps, page 60
Смеситель работает в соответствии с тригонометрическим выражением:

\begin{equation}
\cos(\omega_1 t)\cos(\omega_2 t) = 
	\frac{1}{2}\cos((\omega_1 - \omega_2)t) + \frac{1}{2}\cos((\omega_1 + \omega_2)t)
\label{eq:muxer}
\end{equation}

В нашем случае ${\omega_1}$ соответствует GPS L1 RF частоте 1575.42 МГц, а требуемой частотой является IF частота 2.5 МГц. В соответствии 
с этими условиями локальная частота ${\omega_2}$ должна быть (1575.42 - 2.5) = 1572.92 МГц. Таким образом первая часть формулы
\ref{eq:muxer} будет IF частотой, а вторая будет отфильтрована фильтром 2 (рисунок \ref{pic:hw_receiver}). Формула \ref{eq:muxer} 
предствляет собой упрощенную математическую модель смесителя. В реальных условиях необходимо учитывать потери передачи, нелинейные
искажения, динамические диапазоны, затухание и т.д. В сложной модели необходимо учитывать множество факторов, например
фильтр 2 (рисунок \ref{pic:hw_receiver}) выбирается так, чтобы минимизировать нелинейные искажения, полученные на смесителе.

Упрощенная схема демодуляции представлена на рисунке \ref{pic:signal_conversion}. Входящий сигнал умножается на локальную реплику несущей
и локальную реплику C/A кода. После данных преобразований на выходе приемника находятся биты навигационных сообщений.

\begin{figure}[h]
\begin{center}
	\scalebox{0.5}{\includegraphics[width=1\linewidth]{./pics/signal_conversion.eps}}
\end{center}
\caption{Упрощенная схема демодуляции}
\label{pic:signal_conversion}
\end{figure}

Учитывая вышесказанное можно построить математическую модель преобразования сигнала GPS приемнике,
отображенного на рисунке \ref{pic:hw_receiver} \cite{gps}.
Пусть ${f_{L1}}$ и ${f_{L2}}$ несущие частоты сигналов L1 и L2 от спутника k. Мощности сигналов равны  ${P_C, P_{PL1}}$ и
${P_{L2}}$ для C/A и/или P кодов. Последовательность C/A кода обозначим как ${C^k(t)}$ и P(Y) кода как ${P^k(t)}$.
Навигационные данные обозначим как ${D^k(t)}$. Общий сигнал можно записать в виде:

\begin{eqnarray}
s^k(t) =	\sqrt{2P_C} C^k(t) D^k(t) \cos(2 \pi f_{L1} t) +
		\sqrt{2P_{PL1}} P^k(t) D^k(t) \cos(2 \pi f_{L1} t) + \nonumber \\
		+ \sqrt{2P_{PL2}} P^k(t) D^k(t) \cos(2 \pi f_{L2} t)
\label{eq:gps_signal}
\end{eqnarray}
Как видно, в формуле \ref{eq:gps_signal} на вход RF части поступает сигнал состоящий из C/A и P(Y) 
соствляющих на частотах L1 и L2, а также шум (в формуле не отражен).

После прохождения фильтра, компонента на несущей частоте L2 отфильтровывается и сигнал поступает на смеситель и смещается на
промежуточную частоту (IF).

\begin{eqnarray}
s^k(t) =	\sqrt{2P_C} C^k(t) D^k(t) \cos(\omega_{IF} t) +
		\sqrt{2P_{PL1}} P^k(t) D^k(t) \cos(\omega_{IF} t)
\label{eq:gps_signal_muxer}
\end{eqnarray}

Перед АЦП сигнал фильтруется узкополосным фильтром около частоты C/A кода. В следствии этого, сигнал P(Y) искажается и правая часть
выражения \ref{eq:gps_signal_muxer} превращается в шум ${e(n)}$. Сигнал после обработки АЦП может быть представлен следующим
выражением:

\begin{eqnarray}
s^k(n) =	C^k(n) D^k(n) \cos(\omega_{IF} t) + e(n),
\label{eq:gps_signal_adc}
\end{eqnarray}
с ${n}$ составляющими ${1/f_s}$. Переменная ${n}$ показывает дискретный характер сигнала.

Для получения навигационных данных ${D^k}$ из сигнала, отраженного в выражении \ref{eq:gps_signal_adc}, необходимо его 
перевести в основную полосу частот. Удаление несущей из сигнала выполняется умножением на локальную копию сигнала.
Если локальная копия сигнала совпадает по частоте и фазе с сигналом \ref{eq:gps_signal_adc}, результатом умножения 
будет выражение:
\begin{eqnarray}
s^k(t)	=	C^k(n) D^k(n) \cos(\omega_{IF} n) + e(n) \cos(\omega_{IF} n) = \nonumber \\
	=	-\frac{1}{2}C^k(n) D^k(n) - \frac{1}{2}\cos(2\omega_{IF} n)C^k(n) D^k(n),
\label{eq:gps_signal_lpf}
\end{eqnarray}
где первая часть представляет собой навигационное сообщение, а вторая часть - сигнал на удвоенной частоте несущей, она
отфильтровывается ФНЧ. Сигнал после ФНЧ можно представить в виде:
\begin{eqnarray}
\frac{1}{2}C^k(n) D^k(n).
%\label{eq:gps_signal_ca}
\end{eqnarray}

Следующим шагом является удаление ${C_k(n)}$ C/A кода из сигнала. Это осуществляется с помощью коррелирования 
входящего сигнала с локальной копией C/A кода. Если локальная копия точно повторяет C/A код сигнала, 
выход коррелятора будет представлять:
\begin{eqnarray}
\sum_{n=0}^{N-1} C^k(n) C^k(n) D^k(n) = ND^k(n),
%\label{eq:gps_signal_ca}
\end{eqnarray}
где ${ND^k(n)}$ - навигационное сообщение, представленное N-точками сигнала.

Далее биты преобразуются в навигационные сообщения. В данных навигационных сообщениях
находятся данные, необходимые для определения координат приемника.

%--------------------------------------------------------------------------------
\subsection{Технология точной оценки частоты несущей}
Так как всегда присутствуют сдвиги несущей С/A кода и GPS сигнала, C/A код должен быть быть извлечен из сигнала. Процесс сопровождения 
сигнала "следует" за сигналом и декодирует информацию из навигационных сообщений. Если GPS приемник стационарный, то ожидаемое
изменение частоты, обусловленное движением навигационного спутника, очень низкое (в \cite{tsui} рассчитано максимальное 
значение для стационарного
приемника 0.936 Гц/с, для приемника движущегося с ускорением свободного падения это значение является уже очень существенным 
51.5 Гц/с).
Учитывая это, локальное изменение частоты должно также просиходить с малой скоростью, по этой причине скорость обновления в
локальном ФАПЧ может быть тоже малой. Таким образом, для слежения за GPS сигналом необходимо два ФАПЧ модуля. Один из модулей
используется для слежения за изменением несущей частоты, он называется несущей петлей (carrier loop). Другой модуль 
используется для слежения за С/A кодом, он называется кодовой петлей (code loop).

Наиболее используемыми в программной реализации GPS приемника являются два метода: ФАПЧ (PLL) и блочная подстройка синхронизирующегося
сигнала (block adjustment of synchronizing signall - BASS). BASS метод является немного чувствительным к шуму.

Чтобы начать процесс подстройки частоты несущей, нужно произвести точную оценку несущей входящего сигнала.
Из 1 мс данных можно получить частоту с точностью не превышающей 1 кГц, что является слишком грубой оценкой
для запуска ФАПЧ. Для запуска ФАПЧ частота генерируемого сигнала должна отличаться от частоты входящего
сигнала в пределах нескольких герц.

Основным подходом к точной оценке частоты является подход основанный на разнице фаз. После удаления C/A кода
из входящего сигнала, сигнал становится непрерывным (в данном случае наложение битов информации не предусматривается,
случай с наложением битов будет рассмотрен ниже). Если корреляционный пик в проанализированной 1 мс найден в момент времени ${m}$
и он составляет ${X_m(k)}$, ${k}$ представляет частоту, на которой найден пик. Начальная фаза может быть найдена после ДПФ
как:
\begin{eqnarray}
\theta_m(k) = \tan^{-1}(\frac{Im(x_m(k))}{Re(x_m(k))}),
\label{eq:theta_m}
\end{eqnarray}
где ${Im}$ и ${Re}$ представляют мнимую и реальную части соответственно. Предположим, что во время ${n}$, короткое время спустя после ${m}$,
ДПФ компонента ${X_n(k)}$ 1 мс также является пиком, потому что входящая частота не будет быстро изменяться в течении короткого промежутка 
времени. Начальная фаза угла входящего сигнала во время ${n}$ и частотная компонента ${k}$:
\begin{eqnarray}
\theta_n(k) = \tan^{-1}(\frac{Im(x_n(k))}{Re(x_n(k))}),
\label{eq:theta_n}
\end{eqnarray}

Фазы из уравнений \ref{eq:theta_m} и \ref{eq:theta_n} могут быть использованы для точной оценки частоты:
\begin{eqnarray}
f = \frac{\theta_n(k) - \theta_m(k)}{2\pi(n - m)}
\label{eq:f_fine}
\end{eqnarray}

Из уравнения \ref{eq:f_fine} можно получить значение частоты гораздо более точное, чем значение полученное из ДПФ. Для получения однозначного
значения разность фаз ${\theta_n - \theta_m}$ должна быть меньше ${2\pi}$ \cite{gps}.

В \cite{tsui} предложен алгоритм, который учитывает изменение фазы в следствие изменения значения бита во входящем сигнале.
\begin{enumerate}
\renewcommand{\labelenumi}{\arabic{enumi}.}
    \item Вычислить смещения начала C/A кода и частоту несущей во входном сигнале. Частота несущей может быть найдена с разрешением 1 кГц.
    \item Пик ${X(k)}$ найден на частоте ${k}$, вычислить ДПФ в двух точках. Одна точка на 400 кГц ниже, а другая точка на 400 кГц выше ${k}$ в ${X(k)}$.
    	  Максимум из ${[X(k-400), X(k), X(k+400)]}$ будет новым ${X(k)}$ и используется в ДПФ для точной оценки несущей.
    \item Произвольно выбираются пять миллисекунд относительно начала C/A кода. Из этих данных удаляется C/A код. В этом куске может
          содержаться один сдвиг фазы на ${\pi}$ между двумя любыми 1 мс данных.
    \item Находится ${X_n(k)}$ на входнных данных, где ${n = 1, 2, 3, 4, 5}$. Далее находится фаза из уравнения \ref{eq:theta_m}. Разность может
          быть найдена как:

	\begin{eqnarray}
	\Delta{\theta} = \theta_{n+1} - \theta_n
	\label{eq:Delta_phi}
	\end{eqnarray}

    \item Абсолютное значение разности углов должно быть меньше парога (${2.3\pi/5}$) \cite{tsui}. Если данное условие не выполняется, ${2\pi}$
          может быть добавлено или вычтено из ${\Delta{\theta}}$. Если результат все еще выше порога, ${\pi}$ может быть добавлено или вычтено
	  из ${\Delta{\theta}}$ для учета сдвига фазы на ${\pi}$. Этот результат также сравнивается с порогом ${2.3\pi/5}$. Если результат выше
	  порога, ${2\pi}$ может быть прибавлено или вычтено для получения требуемого результата. После данных манипуляций, конечное значение будет
	  удовлетворять требованиям порога.
    \item Уравнение \ref{eq:f_fine} может быть использовано для точного определения частоты. В 5 мс отрезке будет 4 отрезка по 1 мс для точного
          определения частоты. Среднее значение этих четырех значений будет требуемой частотой. Осреднение необходимо для повышения точности результата.
\end{enumerate}

% Costas loop

Для успешной демодуляции навигационных данных необходимо получить точную локальную копию несущего сигнала. А
также осуществляеть его подстройку с входным сигналом, чтобы исключить влияние допплеровского сдвига. Для этой 
цели можно использовать петлю Костаса.

Цель петли Костаса - попытаться сохранить всю энергию в синфазной компоненте (рисунок \ref{pic:costas}).

\begin{figure}[H]
\begin{center}
	\scalebox{0.9}{\includegraphics[width=1\linewidth]{./pics/costas_loop.eps}}
\end{center}
\caption{Петля Костаса}
\label{pic:costas}
\end{figure}

Для сохраниения энергии в синфазной компоненте необходима цепь обратной связи. Если локальная копия несущей точно выровнена, смеситель в 
синфазной компоненте даст сумму:
\begin{eqnarray}
D^k(n)\cos(\omega_{IF}n + \phi) = \frac{1}{2}D^k(n)\cos(\phi) + \frac{1}{2}\cos(2\omega_{IF}n + \phi),
\label{eq:costas_i}
\end{eqnarray}
где ${\phi}$ - сдвиг фазы между локальной копией несущей и несущей входного сигнала. Смеситель в квадратурной
компоненте даст сумму:
\begin{eqnarray}
D^k(n)\sin(\omega_{IF}n + \phi) = \frac{1}{2}D^k(n)\sin(\phi) + \frac{1}{2}\sin(2\omega_{IF}n + \phi).
\label{eq:costas_q}
\end{eqnarray}
Если оба сигнала прошли через ФНЧ, компоненты с удвоенной частотой будут отфильтрованны и останется:
\begin{eqnarray}
I^k = \frac{1}{2}D^k(n)\cos(\phi), \\
Q^k = \frac{1}{2}D^k(n)\sin(\phi).
\label{eq:costas_iq}
\end{eqnarray}
Для нахождения параметра обратной связи в петле Костаса, можно использовать выражение:
\begin{eqnarray}
\frac{I^k}{Q^k} = \frac{\frac{1}{2}D^k(n)\cos(\phi)}{\frac{1}{2}D^k(n)\sin(\phi)} = \tan(\phi),
\label{eq:costas_descr}
\end{eqnarray}
\begin{eqnarray}
\phi = \tan^{-1}({\frac{Q^k}{I^k}}).
\label{eq:costas_phi}
\end{eqnarray}

Из выражения \ref{eq:costas_phi} видно, что фазовая ошибка равна нулю когда корреляция в квадратурной
компоненте нулевая, а в синфазной компоненте корреляция максимальная. Дескриминатор из выражения \ref{eq:costas_phi}
является наиболее точным дескриминатором цепи Костаса \cite{gps}, но вместе с тем наиболее долгим.

В \cite{gps} представлен сравнительный анализ дескриминаторов. Рисунок \ref{pic:descriminator} из \cite{gps} дает сравнительную
характеристику ответов дескриминаторов цепи Костаса.

\begin{figure}[h]
\begin{center}
	\scalebox{0.8}{\includegraphics[width=1\linewidth]{./pics/descriminator.eps}}
\end{center}
\caption{Сравнительная характеристика дескриминаторов цепи Костаса}
\label{pic:descriminator}
\end{figure}

Из рисунка \ref{pic:descriminator} видно, что выход дескриминатора равен нулю когда смещение фазы 0 или ${\pm{180}^\circ}$.
Это объясняет почему петля Костаса является не чувствительной к сдвигу фазы на ${180^\circ}$ при смене значения бита
в навигационном сообщении.

Более подробно работу петли Костаса можно проиллюстрировать на рисунке \ref{pic:phasor}. 

\begin{figure}[h]
\begin{center}
	\scalebox{0.8}{\includegraphics[width=1\linewidth]{./pics/phasor.eps}}
\end{center}
\caption{Сравнительная характеристика дескриминаторов цепи Костаса}
\label{pic:phasor}
\end{figure}
На данном рисунке векторная сумма ${I^k}$ и ${Q^k}$ изображена как вектор. Если локальная реплика несущей совпадает по фазе с входным сигналом, вектор будет
выровнен с осью ${I}$. Когда сигнал выровнен верно, векторная сумма ${I^k}$ и ${Q^k}$ будет оставаться выровненной с осью ${I}$.
Это свойство гарантирует, что при смене значения бита в навигационном сообщении, вектор на фазовой диаграмме перевернется на
${180^\circ}$ (показано пунктирным вектором). Таким образом при смене цепь Костаса продолжает корректно работать с входящим сигналом.
Это свойство сделало цепь Костаса распространенным выбором для решения задач ФАПЧ в GPS приемниках.

%--------------------------------------------------------------------------------
\subsection{Уточнение фазы расширяющего кода}
Основная задача цепи слежения за расширяющим кодом - отслеживать изменение фазы расширяющего кода во входном
сигнале. Выходом такой цепи является идеально выровненная локальная копия кода. Цепью слежения за изменением
фазы расширяющего кода в GPS является delay locked loop (DLL). Идея лежащая в DLL - это коррелирование
сигнала с тремя репликами кода (рисунок \ref{pic:dll}).

\begin{figure}[h]
\begin{center}
	\scalebox{0.9}{\includegraphics[width=1\linewidth]{./pics/DLL.eps}}
	%\includegraphics[width=80mm,bb=0 0 300 400]{./pics/DLL.pdf}
\end{center}
\caption{Схема DLL с шестью корреляторами}
\label{pic:dll}
\end{figure}

Первым шагом является перевод сигнала на основную полосу частот путем умножения входящего сигнала на локальную реплику
несущей. Следующим шагом является умножение с тремя репликами расширяющего кода (early - ранней, prompt - без сдвига, late - запаздывающей).
Эти реплики сгенерированны со  смещением в 0.5 чипа.
После второго умножения, все выходы интегрируются и дампятся. Выходом этих интеграторов является
значение, которое характеризует степень коррелирования входящего сигнала с локальной репликой расширяющего кода.
Значения выходов сравниваются для определения лучшей корреляции.

\begin{figure}[h]
\begin{center}
	\scalebox{1}{\includegraphics[width=1\linewidth]{./pics/DLL_cor.eps}}
\end{center}
\caption{Пример работы DLL}
\label{pic:DLL_cor}
\end{figure}

На рисунке \ref{pic:DLL_cor} представлен пример работы DLL цепи. Выходы трех корреляторов ${I_E, I_P, I_L}$ сравниваются для определения
максимального значения корреляции. В левой части рисунка \ref{pic:DLL_cor} late код имеет максимальное значение, в данном случае значение
фазы необходимо уменьшить. В правой части рисунка максимальное значение соответствует prompt значению локальной реплики, а early и late
реплики имеют одинаковое значение. В данном случае фаза отслеживается верно.

Преимущество дизайна, отраженного на рисунке \ref{pic:dll}, заключается в том, что он не чувствителен к фазе входящего
сигнала. Если входящий сигнал совпадает по фазе с локальной репликой, то вся энергия будет сосредоточена в 
синфазной части схемы. Если же локальная реплика не совпадает с фазой входящего сигнала, энергия будет
распределена между синфазной и квадратурной частями схемы.

Если скорость цепи слежения за расширяющим кодом не является основным критерием выбора, то следует выбирать
цепь как с синфазной так и с квадратурной составляющей \cite{gps}.

\addcontentsline{toc}{subsection}{Выводы}
\subsection*{Выводы}
В данном разделе были рассмотренны методы слежения за фазой, частотами несущей и C/A кода. Также была рассмотренна математическая модель
GPS приемника. Был обоснован выбор алгоритмов и дано математическое описание преобразований сигнала.

\newpage

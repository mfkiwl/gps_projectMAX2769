\part{Специальная часть}

\section{Сетевая часть}

%\part \chapter) \section)  \subsection
%\subsubsection \paragraph \subparagraph

\subsection{Протокол взаимодействия программного сервера и GPS-Board}
\subsubsection{Общие сведения}
Бинарный протокол связи протокол связи. Клиент посылает 64 бита данных. 8 младших бит - команда и 56 бит - данные. Плата возвращает 8 бит. В случае
удачи - плата возвращает команду (т.е. если послано 0x0000000000000002, то вернется 0x02), в случае неудачи ее инверсный вариант. Если команда не известная
возвращается 0xFF.

\subsubsection{Доступные команды}
\begin{itemize}
\item 0xAA - echo-тест RS232 \\
Команда тестирования соединения с GPS-Board по RS232-порту. В бинарном виде 0xAA является наборо чередующихся нулей и единиц;
\item 0x02 - тестирование SRAM \\ 
Команда тестирования SRAM-чипа;
\end{itemize}

\subsection{Протокол взаимодействия программного сервера и графического клиента}
\subsubsection{Общие сведения}
Протокол взаимодействия графического клиента с платой является достаточно простым текстовым протоколом текстовым протоколом. Шаблоном
команды является:
\begin{center}
command[:len=value]
\end{center}
command - команда \\
len - длинна значения ключа (value), тут не учитываются 2 завершающих символа (см. ниже)\\
value - сам ключ \\*

Cвязка :len=value является не обязательным параметром, о чем говорят квадратные скобки. Каждая команда заканчивается символом конца строки
и символом перевода каретки (0x0d 0x0a). Это важно и это следует учитывать. На каждую успешную команду сервер возвращает ACK, в случае
неудачи возвращается ERR и текст ошибки.

\subsubsection{Доступные команды}
\begin{itemize}
\item HELLO\_GPS\_BOARD \\
Команда инициализации сессии. На данном этапе, сервер обрабатывает только 1 сессию, в той последовательности команд которая приведена в 
данном разделе;
\item RS232\_PORT:010=/dev/ttyS0
Команда установки RS232-порта. При этой команде сервер открывает порт. В случае неудачи возвращается ошибка;
\item TEST\_RS232
Команда тестирования соединения с GPS-Board;
\item TEST\_SRAM
Команда тестирования SRAM-микросхемы памяти на GPS-Board.
\end{itemize}

\section{Hardware-часть}

\subsection{RS232-интерфейс}
\subsubsection{Специфика hardware}
\subsubsection{Контроллер RS232-порта}
\paragraph{TX-модуль}
Модуль передачи данных для RS-232 порта на микросхеме Xilinx
\paragraph{RX-модуль}
вставить КА
\paragraph{Генератор тактовой частоты для RS-232}


\subsection{Бинарный протокол для взаимодействия PC и GPS-Board}
\subsubsection{Общие сведения}
Конечный автомат на Xilinx 
\subsubsection{Доступные команды}
в разделе ПО

\subsection{SRAM - статическая память}
\subsubsection{Специфика hardware}
Частота работы платы 50Mhz=20ns. Длительность цикла чтения/записи из/в SRAM для чипа M5M5V208FP-85L 85ns\\

Цикл операции записи в SRAM:
\begin{tabbing}
Название цикла\qquad\=Обозначение\qquad\=Время\qquad\=Тактов кварца\qquad\=Реальное время\\
\\
выставить адрес \> tsu(A) \> 0ns \> 0 \> 0ns \\
выставить WE/OE \> tdis(W) \> 30ns \> 2 \> 40ns \\
выставить данные \> tsu(D) \> 35ns \> 2 \> 40ns \\
снять WE/OE \> ten(W) \> 5ns \> 1 \> 20ns \\
\\
Итого \> \> \> 5  \> 100ns\\
\end{tabbing}

Цикл операции чтения из SRAM:
\begin{tabbing}
Название цикла\qquad\=Обозначение\qquad\=Время\qquad\=Тактов кварца\qquad\=Реальное время\\
выставить адрес \> ta(A) \> 85ns \> 5 \> 100ns \\
Итого \> \> \> 5 \> 100ns \\
\end{tabbing}

\subsubsection{Контроллер SRAM}

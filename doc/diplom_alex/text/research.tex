\section{ИСЛЕДОВАТЕЛЬСКИЙ РАЗДЕЛ}

% =========================== 1.1
\subsection{Параметры сигнала GPS}
\label{razdel11}
\subsubsection{Частота передачи}
Сигнал GPS состоит из двух частотных компонент: link 1 (L1) и link 2 (L2). Центром частоты L1 является
1575.42 МГц и центром частоты L2 является 1227.6 Мгц. Эти частоты когерентны с частотой 10.23 МГц. Они могут быть
получены как:

\begin{equation}
L1=1575.42\mbox{МГц}=154*10.23\mbox{МГц}
\label{eq:l1_freq}
\end{equation}

\begin{equation}
L2=1227.6\mbox{МГц}=120*10.23\mbox{Мгц}
\label{eq:l2_freq}
\end{equation}

Частоты генерируются с высокой точностью. Сгенерированная частота немного ниже чем 10.23 МГц, для учета 
релятивисткского эффекта. Сдвиг частоты составляет ${-4.567\cdot10^{-3}\mbox{Гц}}$. Он согласован c долей
${-4.4647\cdot10^{-10}}$:

\begin{equation}
-4.567\cdot10^{-3}/10.23\cdot10^{6} = -4.4647\cdot10^{-10}
\end{equation}

По этой причине, используемая частота составляет 10.229999995433 МГц ${(10.23\cdot10^{6} - 4.567\cdot10^{-3})}$,
а не 10.23 МГц. Когда приемник принимает сигнал, он находится
на нужной частоте. Однако, движение спутника и приемника может создавать Допплеровский эффект. Частота Допплеровского
сдвига, создаваемого движением спутник, на частоте L1 составляет примерно ${\pm5\mbox{кГц}}$ \cite{yacenkov, tsui}.

Структура сигнала может поменяться в будущем, но на данный момент L1 частота содержит C/A и P(Y) сигналы, в то
время как L2 содержит только P(Y) сигнал. C/A и P(Y) сигналы на частоте L1 сдвинуты на 90 градусов
относительно друг-друга и могут быть записаны как:

\begin{equation}
S_{L1}=A_p P(t) D(t) cos(2\pi f_1 t + \phi) + A_c C(t) D(t) sin(2\pi f_1 t + \phi),
\label{eq:l2_freq}
\end{equation}

где ${S_{L1}}$ - сигнал на L1 частоте, ${A_p}$ - амплитуда P-кода, ${P(t)=\pm 1}$ представлет фазу P-кода,
${D(t) = \pm 1}$ представляет данные, ${f_1}$ - L1 частота, ${\phi}$ - начальная фаза, ${A_c}$ - амплитуда C/A кода,
${C(t)} = \pm 1$ представляет фазу C/A кода.

\subsubsection{C/A код}
GPS сигнал является фазо-модулированным сигналом с ${\phi = 0, \pi}$. Такой вид модуляции называется бифазной (BPSK).
Скорость изменения фаз соответствует скорости чипа. Форма спектра может быть описана как sinc-функция, с шириной спектра
пропорциональной скорости чипа. 

C/A код бифазно-модулированный сигнал со скоростью чипа 1.023 Мгц. Однако, ширина спектра "от нуля до нуля" составляет
2.046 МГц. Каждый чип имеет длительность 977.5 нс (1/1023МГц). Полный период кода составляет 1023 чипа. Со скоростью
чипа 1.023 МГц, 1023 чипа равны 1мс, таким образом C/A код имеет длинну 1 мс. Этот код повторяется каждую миллисекунду.
Спектр C/A кода представлен на рисунке \ref{pic:ca_spectrum}

Учитывая вышесказанное, чтобы найти начало C/A кода требуется запись продолжительностью 1 мс. Если нет Допплеровского
эффекта в принятом сигнале, тогда 1 мс содержит все 1023 чипа. Разные C/A коды соответствуют разным спутникам.
C/A код относится к семейству Gold кодов.

\begin{figure}
\begin{center}
%\scalebox{0.9}{
\includegraphics[width=1\linewidth]{./pics/ca_spectrum.eps}
%}
\end{center}
\caption{Пример спектра C/A кода}
\label{pic:ca_spectrum}
\end{figure}


\subsubsection{Генерирование C/A кода}
\begin{equation}
G1=1 + x^{3} + x^{10}
\label{eq:g1}
\end{equation}


\begin{equation}
G2=1 + x^{2} + x^{3} + x^{6} + x^{8} + x^{9} + x^{10}
\label{eq:g2}
\end{equation}


\subsubsection{Навигационные сообщения}
Каждый штатно функционирующий навигационный спутник передает навигационное сообщение, содержащее оперативную и неоперативную
навигационную информацию. Эта информация предназначена как для проведения текущих навигационных определений, так и для
планирования будущих сеансов приема \cite{yacenkov}.

Оперативная информация относится к тому спутнику, с борта которого эта информация передается, и содержит следующие данные
\cite{yacenkov, tsui}:
\begin{itemize}
\item координаты и параметры орбиты спутника в фиксированный момент времени (эфемериды);
\item сдвиг шкалы времени спутника относительно земной шкалы;
\item относительный сдвиг несущей частоты излучаемого сигнала от номинального значения;
\item код метки времени, необходимый для синхронизации аппаратуры потребителя.
\end{itemize}

Неоперативная информация относится к СНС в целом и содержит альманах системы:
\begin{itemize}
\item данные о функциональном состоянии всех спутников (альманах состояния);
\item сдвиг шкалы времени каждого спутника относительно системной шкалы (альманах фаз);
\item параметры орбит всех спутников системы (альманах орбит);
\item поправку к шкале времени относительно UTC.
\end{itemize}

\subsubsection{Формат GPS сообщений}

\begin{figure}{H}
\begin{center}
\scalebox{0.5}{\includegraphics[width=1\linewidth]{./pics/gps_data_format.eps}}
\end{center}
\caption{Cтруктура сообщений навигационной системы GPS NAVSTAR}
\label{pic:gps_data_format}
\end{figure}

% ========================== 1.2
\subsection{Обзор решений в области захвата сигналов в системах беспроводной передачи}
\label{razdel12}
\subsubsection{Общие понятия}
Захват данных(data acquisition - DAQ) - это процесс регистрации данных реального мира и преобразования полученных данных в 
числовой вид, который может быть обработан на ПК. Захват данных и системы захвата данных(data acquistion systems - DAS) подразумевает
преобразование аналоговых волн в дискретные числовые значения для последующей обработки \cite{ni_acq}. В системы захвата данных входят:

\begin{itemize}
\item Сенсоры(таймеры и тд), которые конвертируют физические параметры в электрические сигналы;
\item Схемы конвертации сигналов с сенсоров в формы пригодные для конвертирования в цифровой формат;
\item АЦП, которые конвертируют информацию со схем конвертации в цифровой формат.
\end{itemize}

\begin{figure}[H]
\center{\includegraphics[width=1\linewidth]{./pics/acq_scheme.eps}}
\caption{Системы захвата данных}
\label{pic_acq}
\end{figure}

На рисунке \ref{pic_acq} отражена наиболее общая схема системы захвата сигналов. Есть датчики, аппаратное обеспечение,
преобразующее сигналы с сенсоров в сигналы пригодные для подачи на АЦП и есть набор управляющего и анализирующего ПО
на ПК \cite{ni_acq}.

Одной из существенных проблем при разработке устройств для анализа сигналов и непосредственной работе по анализу сигналов
является проблема повторяемости данных \cite{ni_article}. Для примера, при отладке ГЛОНАСС навигатора необходимо 
воспроизвести некоторую ситуацию, однажды полученную при захвате сигнала. Очевидно, что сделать это крайне сложно, потому
что сигнал меняется со временем и воспроизвести ситуацию практически невозможно. Хорошим решением данной задачи является
работа с сохраненным сигналом. В данном случае проблема воспроизведения ситуации сводится к повторной подаче сохраненного
сигнала на вход тестируемого устройства. В данный момент существует два способа получения сигнала для последующего анализа:
генерирование сигнала с заданными характеристиками, запись реального сигнала \cite{ni_article}.

\subsubsection{Портативный анализатор сигналов WiMAX (SeaMAX Mobile WiMAX Analyzer)}
Данный анализатор является IEEE 802.16e-2005 OFDMA PHY Base станцией анализа сигналов. Данная станция 
поддерживает 1.25, 3.5, 7, 8.75, 10, 14, 17.5, 20 и 28 Мгц полосы пропускания и позволяет настроить
частоту скорость захвата сигнала. SeaMAX Mobile предоставляет гибкую систему конфигурирования параметров PHY, таких как
размер БПФ и циклического префикса или выбрать опцию автоматического выбора данных параметров. Импорт IQ сигналов
из .txt файлов. Анализатор позволяет сохранять декодированные данные в шестнадцатеричныйричный или двоичный файл,
визуализировать и анализировать параметры передачи, включая смещение частоты, затухание канала
\cite{seamax_overview, seamax_pdf}.

\begin{figure}[H]
\center{\includegraphics[width=1\linewidth]{./pics/seamax_mono.eps}}
\caption{Схема работы с анализатором сигналов SeaMAX}
\label{pic:seamax}
\end{figure}

На рисунке \ref{pic:seamax} представлена схема работы данным анализатором \cite{seamax_pdf}. Он является частью большого комплекса по работе
с WiMAX сигналом. Сигнал создается на генераторе и подается на тестируюемое устройство (DUT), далее с него снимаются 
данные и анализируются на анализаторе. Анализатор производит захват и обработку данных в режиме реального времени и
предоставляет аналитическую информацию на консоль ПК.

\subsubsection{Vector signal generator NI PXIe-5672}
\label{sec:vsg}
Данное устройство от компании National Instruments осуществляет генерацию сигнала специально для тестирования
GPS приложений. Генерируемый поток GPS-данных семплирован на скорости 1.5 МС/c (I/Q) с диска на скорости 6 Мб/c.
Устройство имеет встроенный HDD и интерфейс для подключения внешнего HDD. Типичная конфигурация с PXIe-5672
представлена на рисунке \ref{pic:ni_system}.

\begin{figure}[H]
\begin{center}
\scalebox{0.5}{\includegraphics[width=1\linewidth]{./pics/ni_system.eps}}
\end{center}
\caption{Система от NI с устройствами PXIe-5672 и PXI-5661}
\label{pic:ni_system}
\end{figure}

Используя GPS Simulation Toolkit, можно создать запись длинной до 12.5 мин (25 фреймов), что является полной длинной
GPS-сообщения \cite{yacenkov, tsui} (рисунок \ref{pic:gps_data_format}). Данные, семплированные со скоростью 6 Мб/с,
займут примерно 7.5 Гб. Можно сохранить запись как на одном, так и на нескольких HDD. Доступные варианты:
\begin{itemize}
\item HDD на контроллере PXI;
\item внешний RAID-контроллер, такой как NI HDD-8263 и HDD-8264;
\item внешний USB 2.0 жесткий диск.
\end{itemize}

Каждая из этих HDD-конфигураций поддерживает скорость записи более чем 20 МБ/с непрерывного потока данных.
Данные конфигурации поддерживают как генерирование, так и запись реального GPS-сигналов.

Так как GPS-приемники используют сообщения спутников для получения информации о альманахах и эфемеридах, эта информация
так же необходима для генерирования GPS-сигнала. Предоставляемые как текстовые файлы, альманахи и эфемериды содержат
данные о положении, высоте, состоянии спутника и орбитах, так же, в процессе генерации сигнала, можно настроить параметры
такие как время, положение (широта, долгота, высота) и скорость с которой движется приемник. Основываясь на данной
информации, toolkit автоматически выбирает 12 спутников, считает доплеровский сдвиг и апроксимированное расстояние до
спутника (pseudorange). После выполнения всех рассчетов, устройство генерирует сигнал.

\subsubsection{Vector signal analyzer NI PXI-5661}
В сценарии записи сигнала может использоваться vector signal analyzer (такой как NI PXI-5661) и записываться данные,
сгенерированные на vector signal generator (таком как NI PXIe-5672 \ref{sec:vsg}). Используя данную сборку, можно
протестировать поведение GPS-приемника при заданном сигнале.

Для каждого типа беспроводной связи требования к ширине полосы пропускания, центральной частоте и требуемое усиление разные.
В случае GPS, основным является захват 2.046Мгц полосы пропускания с центральной частотой 1.57542ГГц. Основанная на
широте полосы пропускания, скорость семплирования должна быть не менее 2.5МС/c (1.25 x 2Мгц). 

Сложным аспектом записи GPS-сигнала является выбор и настройка соответствующей антенны и низкошмуящего уситителя.
Замечено, что на полосковой пассивной антенне пик наблюдается в L1 GPS диапазоне от -120 до -110 дБм
(тесты показали на -116 дБм) \cite{ni_article}. Так как уровень
мощности GPS-сигнала является крайне низким, требуется сильное усиление для того, чтобы vector signal
analyzer моu захватить все полный динамический диапазон сигнала (bandwitdh). Существует несколько путей достижения
соответствующего уровня усиления сигнала. Можно достичь лучших результатов используя активную антенну с NI PXI-5690
предусилителем. С двумя каскадами малошумящих усилителей, с коэффициентом усиления в 30Дб каждый, общее усиление будет
60Дб (30+30). Таким образом, наблюдаемый пик на vector signal analyzer возрастет с -116 до -56дБм \cite{ni_article}.
Пример подобной системы предствлен на рисунке \ref{pic:ni_gps_receiver}.

\begin{figure}[H]
\begin{center}
\scalebox{0.5}{\includegraphics[width=1\linewidth]{./pics/ni_gps_receiver.eps}}
\end{center}
\caption{Реализация GPS-приемника с двумя малошумящими усилителями}
\label{pic:ni_gps_receiver}
\end{figure}

Отметим, что одним из основных элементов системы записывания GPS-сигнала является активная GPS антенна. Активные антенны
производятся с низкошумящим усилителем в едином корпусе полосковой антенны. Эти антенны имеют расброс по питанию от 2.5 до
5В и используют SMA-коннектор. 

\subsubsection{Устройство захвата GPS сигналов softGPS}

% =============== 1.3
\subsection{Общая структура разрабатываемых программно-аппаратных средств}
\label{razdel13}
Типичной архитектурой захвата сигналов с последующей обработкой сигнала является структура с накопителем данных.
Для реализации системы захвата GPS-сигнала я выбираю систему с накоплением данных. Данный выбор обусловлен
ценовой доступностью платформы и ориентацией на лабораторное применение в высших учебных заведениях.
К плюсам данного решения можно отнести возможность обрабатывать одни и те же данные с использованием разных решений.
К примеру, обработка на специализированных математических пакетах таких как MATLAB, обработка на микросхемах с
программируемой логикой (FPGA), реализация на языках высокого уровня. Данное разнообразие предполагает разное
количество прикладываемых усилий и может использоваться как для реализации лабораторных работ (создание
коррелятора на MATLAB), так и написание дипломных работ (реализация коррелятора на микросхемах с программируемой
логикой - FPGA) и даже для научных исследований в области обработки сигналов системы глобальной навигации
Глонасс и Navitel GPS.

В условиях поставленных целей - удешевить устройство, упростить программный и аппаратный код, сделать
комплекс максимально дешевым, было выбрано оборудование для реализации аппаратной платформы захвата "сырого"
сигнала систем спутниковой навигации. В рамках поставленной задачи целесообразно выбрать система хранения данных
на SRAM-микросхеме, интерфейс управления/передачи данных RS232, операционную систему
сервера платы Linux, язык программирования для реализации сервера платы C.

\subsubsection{Носитель данных}
\label{razdel1_sram}
Для реализации носителя данных необходимо выбрать микросхему памяти SRAM. Данные микросхемы достаточно дороги,
но очень просты с точки зрения реализации контроллера памяти. Контроллер памяти должен учитывать только время,
на которое выставляются данные на шину данных и шину адреса, в отличае от технологии DRAM, где требуется реализовывать
интерливинг банков. В то же время используемая память должна быть достаточно быстрой. Данные поступают со
скоростью 7.7Мб/сек (скорость работы GPS-микросхемы чуть более 16Мгц, за такт микросхема выдает 4 бита, 2 такта
составляют байт данных, таким образом данные поступают с частотой чуть более 8МГц). Чтобы реализовать систему
с данными характеристиками скорость доступа к записи у микросхемы памяти должна быть не менее 125нс.
Разработка контроллера SRAM-памяти рассмотрена в разделе \ref{razdel3_sram}

\subsubsection{Интерфейс управления/передачи данных}
\label{razdel1_rs232}
Перед интерфейсом управления и передачи данных ставится несколько задач: простота реализации, достаточная скорость передачи данных с
носителя, возможность реализации необходимого протокола команд. Наиболее распространенными интерфейсами на сегодняшний день
являются: RS232, USB, PCI. Для выполнения задач хорошо подходит интерфейс RS232. Его простота и универсальность позволят
реализовать на данном интерфейсе протокол управления платой, а скорость является достаточной для комфортной передачи 
небольшого объема данных со SRAM-микросхемы. Реализация интерфейса рассмотрена в разделе \ref{razdel3_rs232}.

В рамках интерфейса RS232 необходимо разработать протокол управления платой. Протокол должен являеться
бинарным, длина команд фиксирована.
Данные критерии были выбраны по следующим причинам: бинарный протокол проще обрабатывать на аппаратной части системы,
фиксированная длинна команд так же упрощает разбор команд. 
Подробности разработки протокола рассмотрены в главе \ref{rszdel3_mem}.

\subsubsection{ОС сервера платы}
\label{razdel1_os}
Выбор ОС для реализации сервера разработанной платы практически безальтернативен. Наиболее распространенными бесплатными ОС
на сегодняшний день является семейство операционных систем BSD и ОС Linux. Нами был выбран Linux. ОС с проприетарным кодом Windows
нам не подошла по нескольким причинам: стоимость даже базовой версии сопоставима со стоимостью всей платы, цена средств разработки
для данной ОС достаточно высока. Под ОС Linux мы смогли использовать бесплатный компилятор, отладчики, средство контроля версий
и другое ПО для разработки.

\subsubsection{ПО управления сервера платы}
\label{razdel1_sw}
Разработанное мной ПО для управления платой является комплексным средством. Основными задачами при его разработке ставятся возможность
расширения и простота использования конечными клиентом. Реализуя продукт для использования в университетах мы должны создать
простое и бесплатное программное обеспечение, стоимость внедрения которого была бы не большой.
Сервер платы должен быть гибким - полученные с платы данные могут выкладываться на любой носитель и даже записываться на CD-диски.
Так же он должен быть снабжен интерфейсом управления с графическим интерфейсом и иметь интуитивно понятный файл конфигурации.
Более подробная информация о сервере платы содержится в \ref{razdel3_sw}.

\newpage

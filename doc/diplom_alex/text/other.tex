\section{ДРУГИЕ РАЗДЕЛЫ ПРОЕКТА}

%==========================================================================================

\subsection{Руководство пользователя по работе с приложением}

Для запуска в фоне используется сторонняя программа screen. Порядок работы с программой screen:

\begin{enumerate}
\renewcommand{\labelenumi}{\arabic{enumi}.}
\item запуск программы screen из терминала.
\item запуск board daemon, путем запуска скрипта \textbf{./start}.
\item отсоединение от screen-терминала с помощью команды \textbf{Ctrl+a+d}.
\end{enumerate}
О успешном старте board daemon свидетельствует надпись \textbf{[rs232\_process] Process...}

Если пользователю нужно создать дамп по запросу, ему необходимо выполнить следующую цепочку операций:
\begin{enumerate}
\renewcommand{\labelenumi}{\arabic{enumi}.}
\item c помощью команды \textbf{ps -waux | grep board\_daemon} узнать pid процесса board daemon
\item выполнить команду \textbf{kill -USR1 pid}, где pid - номер процесса, полученный на предыдущем шаге 
\end{enumerate}


%==========================================================================================

\subsection{Руководство программиста}
Клонирование исходных текстов проекта осуществляется с помощью системы контроля версий GIT.  Для клонирования
необходимо иметь доступ в интернет и установленный GIT. Клонирование осуществляется командой \textbf{gitclone http://github.com/venik/gps\_project.git}.
После этого на компьютере программиста появится папка gps\_project в которой будет находится последняя версия исходных кодов программного продукта.
В дереве исходного кода существуют папки doc, src, mngmt. 

Папка src содержит все исходные коды проекта. В ней содержатся папки hardware и software. В папке software находятся все программные модули проекта:
сервер платы, парсер конфигурационных файлов, исходные коды ФАПЧ для Matlab. В папке hardware находятся исходные коды модулей для ПЛИС микросхемы Xilinx:
исходный код арбитра платы, контроллера SRAM-памяти, модуля тестирования SRAM-памяти, модуля сохранения полученного сигнала, модуля программирования GPS
микросхемы, контроллер RS-232 интерфейса, модуль парсера двоичного протокола. \\
hardware/arbiter.vhd - арбитр платы выполняет основную управляющую функцию. \\
hardware/rs-232/rs232\_rx\_new.vhd, hardware/rs-232/rs232\_tx\_new.vhd - реализация RS-232 интерфейса. \\
hardware/sram\_M5M5V208FP-85L/sram.vhd - реализация контроллера SRAM-памяти. \\
hardware/mem\_test/test\_sram.vhd - модуль тестирования SRAM-памяти. \\
hardware/gps/gps\_main.vhd, hardware/gps/gps\_serial.vhd - модуль программирования режимов GPS микросхемы, реализация последовательного 
	интерфейса программирования GPS-микросхемы. \\
hardware/top\_level/top\_level.vhd - реализация мультиплексной шины адреса и данных SRAM-микросхемы. \\
src/hardware/board - файлы разводки платы. \\
software/board\_daemon - реализация сервера платы. \\
software/include - заголовочные файлы парсера конфигурационных файлов, заголовочные файлы протоколов. \\
software/manage\_scripts - исходные коды публикации дампов в публичные источники.
software/gpstool - исходные коды для среды Matlab используемые в обработке данных GNSS. \\
software/rs232\_client - исходные коды отладочного клиента платы. \\


В папке mngmt находятся скрипты управления деревом исходного кода проекта.

В папке doc находится документация на плату и документация на сервисное ПО.

%==========================================================================================

\subsection{Руководство системного программиста}
Для использования данных программных средств на компьютере должна быть установлена операционная система Gentoo Linux для сервера платы и MS Windows XP с 
математическим пакетом Matlab для модуля сопровождения сигнала. 
Для установки и работы с программными средствами обозначим минимальные требования к конфигурации компьютера: процессор с тактовой частотой 1 ГГц, память 1024 Мб,
жёсткий диск объёмом 10 Гб, объём видеопамяти 32 Мб. 

Board daemon требует прав root или suid бита для исполнения(требуется для открытия RS-232 порта). Запуск демона осуществляется из консоли командой
\begin{center}
\textbf{./board\_daemon -c default\_cfg}
\end{center}
где через -c ключ указывается конфигурационный файл. В конфигурационном файле указывается режим работы GPS-микросхемы, скрипт публикации дампа и имя RS-232 порта.
Пример конфигурационного файла представлен ниже:

\hrulefill \\
    \# GPS Registers \\
    addr0=a2939a3 \\
    addr1=855028c \\
    addr2=0x6aff1dc \\
    addr3=0x9ec0008 \\
    addr4=0x0c00080 \\
    addr5=0x8000070 \\
    addr6=0x8000000 \\ 
    addr7=0x10061b2 \\
    addr8=0x1e0f401 \\
    addr9=0x14c0402 \\
\\
    \# GUI-server thread \\
    gui\_tcpport=1234 \\
\\
    \# Board-server thread \\
    rs232\_portname=/dev/ttyUSB0 \\
    upload\_script=../manage\_scripts/store\_flush.sh

\hrulefill

Через addrX задается значение для программирование регистра X (X = 0..9). Так же задается порт для подключения графического конфигуратора,
имя RS-232 порта и имя скрипта публикации дампа.

\newpage
